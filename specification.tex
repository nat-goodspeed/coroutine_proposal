\abschnitt{Technical Specification}

\subsubsection*{pull\_coroutine:}
\cppf{pull_coroutine.cpp}

\paragraph*{(constructor):}
constructs new coroutine\\

\begin{tabular}{ l l }
    \midrule

    \cpp{coroutine();} & (1)\\

    \midrule

    \cpp{coroutine(Function&& fn);} & (2)\\

    \midrule

    \cpp{coroutine(Function&& fn, Args... args);} & (3)\\

    \midrule

    \cpp{coroutine(coroutine&& other);} & (4)\\

    \midrule

    \cpp{coroutine(const coroutine& other) = delete;} & (5)\\

    \midrule
\end{tabular}

\begin{description}
    \item[1)] creates a \coro which does not represent a context of execution
    \item[2)] creates a \coro object and associates it with a execution
              context
    \item[3)] creates a \coro object and associates it with a context of
              execution and passes all arguments args to \corofunction fn
    \item[4)] move constructor, constructs a \coro object to represent a
              context of execution that was represented by other, after this
              call other no longer represents a coroutine
    \item[5)] copy constructor is deleted; coroutine are not copyable
\end{description}

{\bf Parameters}
\begin{description}
    \item[other]   another coroutine object to construct this coroutine object with
    \item[fn]      function to execute in the new coroutine
    \item[args...] arguments to pass to the new function
\end{description}

{\bf Exceptions}
\begin{description}
    \item[1), 4)] noexcept specification: \cpp{noexcept}
    \item[2), 3)] \cpp{std::system_error} if the coroutine could not be started
                  - the exception may represent a implementation-specific error
                  condition; re-throw user defined exceptions from \corofunction
\end{description}

{\bf Notes}
The arguments to the \corofunction are copied by value. If a reference argument
needs to be passed to the thread function, it has to be wrapped (e.g. with
\cpp{std::ref} or \cpp{std::ref}).\\
\newline
Return values from the \corofunction are accessible via \coroget. If the
function throws an exception, this exception is re-thrown if the caller returns from
\coroop.\\

{\bf Example}
\cppf{fibonacci.cpp}

\paragraph*{(destructor):}
destructs a coroutine\\

\begin{tabular}{ l l }
    \midrule

    \cpp{\~coroutine();} & (1)\\

    \midrule
\end{tabular}

\begin{description}
    \item[1)] destroys a \coro, if associated with a context of execution, then
              the context of execution is destroyed too (for instance the
              associated stack)\\
\end{description}

\paragraph*{operator=:}
moves the coroutine object\\

\begin{tabular}{ l l }
    \midrule

    \cpp{coroutine & operator=(coroutine&& other);} & (1)\\

    \midrule
\end{tabular}

\begin{description}
    \item[1)] assigns the state of other to *this using move semantics
\end{description}

{\bf Parameters}
\begin{description}
    \item[other]   another coroutine object to assign to this coroutine object
\end{description}

{\bf Return value}
\begin{description}
    \item[*this]
\end{description}

{\bf Exceptions}
\begin{description}
    \item[1)] noexcept specification: \cpp{noexcept}
\end{description}

\paragraph*{operator bool:}
indicates if context of execution is still valid and a return value can be retrieved
or \corofunction has finished\\

\begin{tabular}{ l l }
    \midrule

    \cpp{operator bool();} & (1)\\

    \midrule
\end{tabular}

\begin{description}
    \item[1)] evaluates to true if coroutine has a return value and is not complete
\end{description}

{\bf Exceptions}
\begin{description}
    \item[1)] noexcept specification: \cpp{noexcept}
\end{description}

\paragraph*{operator():}
jump context of execution\\

\begin{tabular}{ l l }
    \midrule

    \cpp{coroutine & operator()(Args... args);} & (1)\\

    \midrule
\end{tabular}

\begin{description}
    \item[1)] switches the context of execution and transfers args to
              \corofunction and makes them accessible by \coroget
\end{description}

{\bf Parameters}
\begin{description}
    \item[args...] arguments to pass to the \corofunction
\end{description}

{\bf Return value}
\begin{description}
    \item[coroutine\&] reference to \cpp{*this}
\end{description}

{\bf Exceptions}
\begin{description}
    \item[1)] \cpp{std::system_error} if the coroutine could not be jumped
              - the exception may represent a implementation-specific error
              condition; re-throw user-defined exceptions from \corofunction
\end{description}

\paragraph*{get:}
accesses the return value from \corofunction\\

\begin{tabular}{ l l }
    \midrule

    \cpp{R get();} & (1)\\

    \midrule
\end{tabular}

\begin{description}
    \item[1)] access values returned from \corofunction
\end{description}

{\bf Return value}
\begin{description}
    \item[R] return type is defined by coroutines template argument
\end{description}


\paragraph*{swap:}
swaps two coroutine objects\\

\begin{tabular}{ l l }
    \midrule

    \cpp{void swap(coroutine& other);} & (1)\\

    \midrule
\end{tabular}

\begin{description}
    \item[1)] exchanges the underlying context of execution of two coroutine objects
\end{description}

{\bf Exceptions}
\begin{description}
    \item[1)] noexcept specification: \cpp{noexcept}
\end{description}


\subsubsection*{push\_coroutine:}
\cppf{push_coroutine.cpp}

\paragraph*{(constructor):}
constructs new coroutine\\

\begin{tabular}{ l l }
    \midrule

    \cpp{coroutine();} & (1)\\

    \midrule

    \cpp{coroutine(Function&& fn);} & (2)\\

    \midrule

    \cpp{coroutine(Function&& fn, Args... args);} & (3)\\

    \midrule

    \cpp{coroutine(coroutine&& other);} & (4)\\

    \midrule

    \cpp{coroutine(const coroutine& other) = delete;} & (5)\\

    \midrule
\end{tabular}

\begin{description}
    \item[1)] creates a \coro which does not represent a context of execution
    \item[2)] creates a \coro object and associates it with a execution
              context
    \item[3)] creates a \coro object and associates it with a context of
              execution and passes all arguments args to \corofunction fn
    \item[4)] move constructor, constructs a \coro object to represent a
              context of execution that was represented by other, after this
              call other no longer represents a coroutine
    \item[5)] copy constructor is deleted; coroutine are not copyable
\end{description}

{\bf Parameters}
\begin{description}
    \item[other]   another coroutine object to construct this coroutine object with
    \item[fn]      function to execute in the new coroutine
    \item[args...] arguments to pass to the new function
\end{description}

{\bf Exceptions}
\begin{description}
    \item[1), 4)] noexcept specification: \cpp{noexcept}
    \item[2), 3)] \cpp{std::system_error} if the coroutine could not be started
                  - the exception may represent a implementation-specific error
                  condition; re-throw user defined exceptions from \corofunction
\end{description}

{\bf Notes}
The arguments to the \corofunction are copied by value. If a reference argument
needs to be passed to the thread function, it has to be wrapped (e.g. with
\cpp{std::ref} or \cpp{std::ref}).\\
\newline
Return values from the \corofunction are accessible via \coroget. If the
function throws an exception, this exception is re-thrown if the caller returns from
\coroop.\\

{\bf Example}
\cppf{fibonacci.cpp}

\paragraph*{(destructor):}
destructs a coroutine\\

\begin{tabular}{ l l }
    \midrule

    \cpp{\~coroutine();} & (1)\\

    \midrule
\end{tabular}

\begin{description}
    \item[1)] destroys a \coro, if associated with a context of execution, then
              the context of execution is destroyed too (for instance the
              associated stack)
\end{description}

\paragraph*{operator=:}
moves the coroutine object\\

\begin{tabular}{ l l }
    \midrule

    \cpp{coroutine & operator=(coroutine&& other);} & (1)\\

    \midrule
\end{tabular}

\begin{description}
    \item[1)] assigns the state of other to *this using move semantics
\end{description}

{\bf Parameters}
\begin{description}
    \item[other]   another coroutine object to assign to this coroutine object
\end{description}

{\bf Return value}
\begin{description}
    \item[*this]
\end{description}

{\bf Exceptions}
\begin{description}
    \item[1)] noexcept specification: \cpp{noexcept}
\end{description}

\paragraph*{operator bool:}
indicates if context of execution is still valid and a return value can be retrieved
or \corofunction has finished\\

\begin{tabular}{ l l }
    \midrule

    \cpp{operator bool();} & (1)\\

    \midrule
\end{tabular}

\begin{description}
    \item[1)] evaluates to true if coroutine has a return value and is not complete
\end{description}

{\bf Exceptions}
\begin{description}
    \item[1)] noexcept specification: \cpp{noexcept}
\end{description}

\paragraph*{operator():}
jump context of execution\\

\begin{tabular}{ l l }
    \midrule

    \cpp{coroutine & operator()(Args... args);} & (1)\\

    \midrule
\end{tabular}

\begin{description}
    \item[1)] switches the context of execution and transfers args to
              \corofunction and makes them accessible by \coroget
\end{description}

{\bf Parameters}
\begin{description}
    \item[args...] arguments to pass to the \corofunction
\end{description}

{\bf Return value}
\begin{description}
    \item[coroutine\&] reference to \cpp{*this}
\end{description}

{\bf Exceptions}
\begin{description}
    \item[1)] \cpp{std::system_error} if the coroutine could not be jumped
              - the exception may represent a implementation-specific error
              condition; re-throw user-defined exceptions from \corofunction
\end{description}

\paragraph*{get:}
accesses the return value from \corofunction\\

\begin{tabular}{ l l }
    \midrule

    \cpp{R get();} & (1)\\

    \midrule
\end{tabular}

\begin{description}
    \item[1)] access values returned from \corofunction
\end{description}

{\bf Return value}
\begin{description}
    \item[R] return type is defined by coroutines template argument
\end{description}


\paragraph*{swap:}
swaps two coroutine objects\\

\begin{tabular}{ l l }
    \midrule

    \cpp{void swap(coroutine& other);} & (1)\\

    \midrule
\end{tabular}

\begin{description}
    \item[1)] exchanges the underlying context of execution of two coroutine objects
\end{description}

{\bf Exceptions}
\begin{description}
    \item[1)] noexcept specification: \cpp{noexcept}
\end{description}
