\abschnitt{Discussion}
\uabschnitt{Implications of \textit{stackless}:}
Resumable functions and resumable lambdas are stackless because the instances of
those types share the same stack (application/main stack) with each other.\\
This implies that no stack frame must remain on the stack at context switch -
otherwise the stack frames get corrupted (overwritten by other code).\\
Suspention from nested functions is not possible!\\
Therefore it is not possible to use resumable functions/lambdas together with
code which requires suspenting the execution context from nested functions.
This would require that the complete code has to be transformed into resumable
functions - which is impossible for closed source code or a too large or complex
code base.

\uabschnitt{Implications of \textit{stackful}:}
In the context of stackful coroutines each context has its own stack,
without interfering with each other. Therefore stack frames of sub-routines can
remain on the stack and supention from nested functions is supported.\\
Tge code executed in a stakfull execution context can be arbitrary and does not
need any modification (in contrast to resumable functions/lambdas if suspention
from nested call stack is required). The appendix\ref{appendix} contains an
example of an recursive descent parser with invertion of control.

\uabschnitt{Activation frame:}
Resumable functions as well as resumable lambdas require storage for control
structures, local variables and resume point (content of certain CPU registers).
The required size will be determined by the compiler and the storage is
allocated on the heap/freestore - such called activation frame.\\
A stackful execution context might allocate its controll structures on top of
its own stack. Local variables use stack space as ordinary fucntions do.
The resume point (CPU registers) are pushed to/poped from at the memory (stack)
where the stack-pointer points to.\\
If the compiler can analyse the code executed in the stackful execution context,
than the stackful context switching shows the same memory requirements as
resumable functions/lambdas.\\
\newline
The statement that stackless coroutines (resumable functions/lambdas) have
always a lower memory requirement than stackful coroutines is not true
\cppf{N4134/fib.cpp}
As shown in the example above, the function body is simple (no sub-routines
called). The compiler determines the required size for the activation frame and
allocates it.\\
In the case of stackful coroutines for generating the sequence of fibonacci
numbers (example below), the required memory could be estemated (library only
solution) or calculated by the compiler (support by the compiler through
language extension).
\cppf{N3985/fib.cpp}
As shown, for simple function bodies stackless and stackful coroutines have
similar memory requirements.\\

\uabschnitt{Calling sub-routines:}
What happens if sub-routines (maybe compiled, from third-party vendors) are
invoked (for instance recursive)? In the context of stackful execution context
the user has to estimate the required stack space or use a stack that grows on
demand (linked stacks). Suspending from the nested call stack is still
possible.\\
On the other side resumable functions/lambdas share the same stack (suspending
from nested functions is not possible). If the stack of the application
(contains stack frame from function main()) is used, the size of the stack grows
on demand. But what happens if those resumable functions/lambdas are executed
inside a thread? The stack size of a thread is fix - usually 1MB (Windows) till
2MB (Linux) but on other platforms 64KB (HP-UX/PA)/256KB (HP-UX/Itanium).\\
If the called sequence of nested sub-routines exceeds the size of the stack the
resumable function/lambda will fail.\\
\newline
Because of this reasons the author beleives that the argument that stackless
resumable functions/lambdas are more space efficient applies not very well.

\uabschnitt{Resume/suspend operations:}
Resumable functions/lambdas provide two operations for context switching. One
operation to resume (for instance \rlop) and one
operation to supend the context (for instance keywork \yield inside the function
body). But this introduced a tight coupeling between invoker and resumable
function/lambda - e.g. a resumable function/lambda can only jump back to the
point in the code where it was invoked.\\
This is sufficient in the case of generators but in the context of cooperative
multitasking it is inefficient!\\
In contrast to resumable functions/lambdas the proposed stackful execution
context (class \ectx) provides only one operation to resume/suspend
the context (\ectxop). The execution control is directly transferred from one
execution context to another - no need to jump back to the caller. This enables
a much more efficient implementation of cooperative multitasking (no additional
context switch back to caller) as well as the implementation of generators.

\uabschnitt{Passing data:}
Because of the asymmmetric resume/suspend operations of N4134 and N4244 both
proposals apply well to generator-examples, e.g. returning data from the
resumable function/lambda u(for isntance using \yield).\\
Passing data into the body of resumable functions/lambdas is not intuitive -
instead verhicles like \cpp{channel<>} (N4134)
\cppf{N4134/channel.cpp}
or \cpp{template< typename T > T * resumable_lambda::wanted()} (N4244)
\cppf{N4244/wanted.cpp}
have to be used.\\


\uabschnitt{Summary stackless:}
\begin{itemize}
    \item   for each resumable function/lambda an activation frame is allocated
    \item   activation frame and stack used for function frames a separat
    \item   stack (used to allocated function frames on it) is shared between
            resumable functions/lambdas
    \item   asymmetric context switching -> tight coupeling between caller and
            invoker (less efficient in the context of cooperative multitasking)
    \item   returning data from resumable function/lambda via \yield etc.
    \item   no intuitive passing data into body of resumable function/lambda
\end{itemize}

\uabschnitt{Summary stackful:}
\begin{itemize}
    \item   each stackful execution context owns a stack (fixed size or growing
            on demand)
    \item   activation frame and stack (for function frmaes) are the same; no
            distinction
    \item   stack is not shared between execution contexts
    \item   symmetric context switching -> no coupeling between caller and
            invoker (more efficient in the context of cooperative multitasking)
    \item   no tansfer of data between context switches (clean API; data transport
            must be organized from sorounding code)
\end{itemize}
