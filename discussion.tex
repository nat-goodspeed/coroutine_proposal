\abschnitt{Discussion}
\uabschnitt{Memory requirements}
In the case of a simple toplevel context-function, stackless and stackful coroutines
have the same memory requirements for their activation records.\\
\newline
As described in N4134, the compiler analyses the body of the toplevel context
function and determines the required size and allocates the activation record on
the heap.
\cppf{N4134/fibonacci}
In the example of calculating Fibonacci numbers using a resumable function, the
compiler reserves space in the activation record for local variables \emph{n},
\emph{a}, \emph{b} and \emph{next}. Those variables are not accessed via the
stack pointer. The processor stack is not used and the stack pointer is not
changed\footnote{Depending on the calling convention, typical \emph{x86} code
stores parameters and return address on the stack. This must be cleaned off by
the time control is returned to the caller. Other architectures/ABIs do
not have this requirement}.\\
\newline
Each stackful coroutine owns a side stack (assigned to the stack pointer). The
activation record of the toplevel context function is stored on the side
stack, thus the original stack remains unchanged.
\cppf{N4397/fibonacci}
In the case of calculating Fibonacci numbers using a stackful execution
context, the compiler stores the local variables \emph{n}, \emph{a},
\emph{b} and \emph{next} on the side stack. The local variables are accessed
via the stack pointer.\\
\newline
The memory requirements for both types of coroutines are equal.

\uabschnitt{Calling subroutines}
The advantage of stackless coroutines is that they reuse the same linear
processor stack for stack frames for called subroutines. The advantage of
stackful context switching is that it permits {\bfseries suspending from
nested calls}.\\
\newline
If a traditional function (not another resumable function) is invoked inside
the body of a resumable function, then the activation record belonging to the
traditional function is allocated on the processor stack (so it is called a
stack frame). As a consequence, stack frames of called functions must be
removed from the processor stack before the resumable function yields back to
its caller.\\
In other words: the calling convention of the ABI dictates that, after the
resumable function returns (suspends), the stack pointer contains the same
address as before the resumable function was entered.\\
Hence a yield from nested call is {\bfseries not permitted} -- unless every
called function down to the yield point is also a resumable function.\\
The benefit of stackless coroutines consists in reusing the processor stack
for called subroutines: no separate stack memory need be allocated.\\
\newline
Of course even a stackless resumable function might fail if its called
functions exhaust the available stack.\\
\newline
In stackful context switching, each execution context owns a
distinct side stack which is assigned to the stack pointer (thus the stack
pointer must be exchanged during each context switch).\\
All activation records (stack frames) of subroutines are placed on the side
stack. Hence each stackful execution context requires enough memory to hold
the stack frames of the longest call chain of subroutines. Therefore, to
support calling subroutines, stackful context switching has a higher memory
footprint than resumable functions.\\
On the other hand, it is beneficial to use side stacks because the stack
frames of active subroutines remain intact while the execution context is
suspended. This is the reason why stackful context switching permits
{\bfseries yielding from nested calls}.
\cppf{N4397/simple}
Variable \emph{l1} is passed to stackful execution context \emph{l2} via
the capture list of \emph{l2}. The stack frames of \cpp{std::printf()} are
allocated on the side stack owned by \emph{l2}.

\uabschnitt{Asymmetric vs. symmetric}
As a building block for user-mode threads, symmetric control transfer is more
efficient than the asymmetric mechanism.\\
\newline
Resumable functions (as proposed in N4134\cite{N4134}) provide two operations
for context switching: asymmetric control transfer. The caller and the
resumable function are coupled, that is, a resumable function can only jump back
to the point in the code where it was entered.
\graphc{asymm}
For {\bfseries{N}} resumable functions {\bfseries{2N}} context switches are
required.\\
\newline
This is sufficient in the case of generators, but in the context of cooperative
multitasking it is inefficient.
In contrast to resumable functions, the proposed stackful execution context
(\ectx) provides only one operation to resume/suspend the context (\ectxop).
Control is directly transferred from one execution context to
another (symmetric control transfer) - no jump back to the caller. In addition
to supporting generators, this enables an efficient implementation of
cooperative multitasking: no additional context switch back to caller,
direct context switch to next task.\\
The next execution context must be explicitly specified.\\
\newline
\graphc{symm}
Resuming {\bfseries{N}} instances of \ectx takes {\bfseries{N+1}} context
switches.

\uabschnitt{Passing data}
Because of the asymmetric resume/suspend operations of N4134, the proposal
applies well to generator examples, e.g. returning data from the resumable
function.\\
\newline
Passing data into the body of resumable functions (N4123) requires helper
classes like \channel.
\cppf{N4134/channel}
In this case, the way data are passed into the body is not intuitive and
introduces some problems.\\
\newline
Experience with \cpp{execution_context} from Boost.Context\cite{bcontext}
turned up a common pattern: to pass a lambda to the \cpp{execution_context},
using its capture list to transfer data into and to return data from the
body. Parameters (input/output) are accessed via captured references or
pointers.
\cppf{N4397/param}
Class \cpp{X} (rudimentary coroutine) demonstrates how input and output
parameters are transferred between contexts. Member variable
\emph{callee\_} represents a new execution context and captures the
\emph{this}-pointer of \cpp{X}. The body converts a integer variable (input)
into a string (output). Any exception thrown by the conversion is transported
via member variable \emph{exptr\_}.

\uabschnitt{Stack strategies}
For stackful coroutines two strategies are typical: a contiguous, fixed-size
stack (as used by threads), or a linked stack (grows on demand).\\
The advantage of a fixed-size stack is the fast allocation/deallocation of 
activation records. A disadvantage is that the required stacksize has to
be guessed.\\
The benefit of using a linked stack is that only the initial size of the stack
is required. The stack itself grows on demand, by means of an overflow handler.
The performance penalty is low. The disadvantage is that
code executed inside a stackful coroutine must be rebuilt for this
purpose. In the case of GCC's split stacks, special compiler/linker flags must
be specified - no changes to source code are required.\\
When calling a library function not compiled for linked stacks (expecting a
traditional contiguous stack), GCC's implementation uses link-time code
generation to change the instructions in the caller. The effect is that a
reasonably large contiguous stack chunk is temporarily linked in to handle the
deepest expected chain of traditional function stack frames.
