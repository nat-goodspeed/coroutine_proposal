\abschnitt{Discussion}
\uabschnitt{Implications of \textit{stackless}}\label{stackless}
Resumable functions/lambdas are characterized as stackless because the instances
of those types share the same stack (application/thread stack) with each other.\\
This implies that all stack frames, allocated inside the body of the resumable
function/lambda, have to be removed from the stack before the resumable
function/lambda is suspended - otherwise the stack frames get corrupted
(overwritten by other code) - yielding from nested functions is not possible!\\
Code like shown in the appendix (recursive descent parser\ref{appendix}) is not
possible without transforming all functions in the call chain into resumable
functions - which is impossible for closed source code or for a too
large/complex code base.

\uabschnitt{Implications of \textit{stackful}}
Stackful context switching requires that each context owns a separate stack.
As a consequence stack frames of sub-routines remain intact while the execution
context gets suspended from nested functions.\\
In contrast to resumable functions, the usage of the stack inside a execution
context does not differ from using the application stack in the main context
(functions called from function \main). For instance local variables are saved
on the stack while resumable functions store local variables in the heap
allocated activation frame.\\
In contrast to resumable functions, the code does not require a modification if
the execution context has to be suspended from nested call stack.

\uabschnitt{Activation frame}
Resumable functions as well as resumable lambdas require storage for local
variables and the control block. The required size is determined by the compiler
and the storage is allocated by the compiler.\\
\cppf{N4134/fib.cpp}
A stackful execution context allocates its controll structures on top of its own
stack (placement new). The control block is pushed to/poped from the memory
(stack) where the stack-pointer points to.\\
If the compiler can analyse the code executed in the stackful execution context,
than the stackful context switching shows the same memory requirements as
resumable functions/lambdas.\\
\newline
For simple function bodies, stackless and stackful coroutines have similar
memory requirements.\\
In the case of resumable functions the compiler determines the required size for
the activation frame (containing space for local variables and control block).\\
In the case of stackful coroutines (using coroutines from N3985), the required
memory has to be estimated (library only solution) or calculated by the compiler
(language extension), but the size of the stack would be similar as the size
of the activation frame.
\cppf{fib.cpp}
Higher level frameworks based on \ectx can realize more sophisticated APIs
(like range based for etc.) - as an example the API of N3985\cite{N3985} is
implemented in boost.coroutine2\cite{bcoroutine2} based on
\cpp{execution_context}.
\cppf{N3985/fib.cpp}

\uabschnitt{Calling sub-routines}
What happens if sub-routines (maybe compiled, from third-party vendors) are
recursive called? In the context of stackful execution context
the user has to guess the required stack space or use a stack that grows on
demand (linked stacks). Suspending from the nested call stack is still
possible.\\
\newline
On the other side, resumable functions/lambdas share the same stack (suspending
from nested functions is not possible). If the resumable function/lambda is
called from the main context (function \main) the application stack is used.\\
If the resumable functions/lambda is invoked inside a thread, the stack size is
fix - usually 1MB (Windows) till 2MB (Linux), but on other platforms smaller
stacks are used (64KB on HP-UX/PA and 256KB on HP-UX/Itanium).\\
If the called sequence of nested sub-routines exceeds the size of the stack the
resumable function/lambda will fail.\\
\newline
Using stackless resumable functions/lambdas requires to take care about the
available stack space.\\
\newline
The benefit of stackless resumable functions/lambdas consists in sharing the
stack, e.g. the space used by the sub-routines is shared between the resumable
functions/lambdas. Allocating this memory for each resumable function/lambda is
not necessary.

\uabschnitt{Resume/suspend operations}
Resumable functions/lambdas provide two operations for context switching. One
operation to resume (\rlop) and one operation to suspend the context (\yield
inside the function body). The caller and the resumable function/lambda are
coupled - e.g. a resumable function/lambda can only jump back to the point in
the code where it was resumed.\\
\newline
For N resumable functions/lambdas 2*N context switches are required.\\
This is sufficient in the case of generators but in the context of cooperative
multitasking it is inefficient.
\imgc{asymm.pdf}
In contrast to resumable functions/lambdas the proposed stackful execution
context (\ectx) provides only one operation to resume/suspend the context
(\ectxop). The execution control is directly transferred from one
execution context to another - no jump back to the caller. This enables a much
more efficient implementation of cooperative multitasking (no additional context
switch back to caller) as well as the implementation of generators. As an
conmsequence, the next execution context has to be explicitly specified.\\
\newline
Resuming N instances of \ectx takes N+1 context switches.
\imgc{symm.pdf}

\uabschnitt{Passing data}
Because of the asymmetric resume/suspend operations of N4134 and N4244 both
proposals apply well to generator-examples, e.g. returning data from the
resumable function/lambda (for instance using \yield).\\
Passing data into the body of resumable functions (N4123) requires helper
classes like \channel.
\cppf{N4134/channel.cpp}
N4244 resumable lambda specifies the member function \rlwanted.
\cppf{N4244/wanted.cpp}
In both cases the way how data have to be passed into the body is not intuitive
and provides some problems how returned values should be accessed. For instance
multiple parameters must be transported with an container (\tuple) or via
multiple invocations of \rlwanted.\\
\newline
An alternative would be not directly support parameter passing in the API.\\
During the use of \cpp{execution_context} from boost.context\cite{bcontext} a
common pattern was to pass lambdas to the execution context and use the capture
list to pass data into and return data from the body. Exceptions are transferred
via a reference of \excpt and parameters (input/output) are accessed via
captured addresses (reference or pointer).
\cppf{param.cpp}

\uabschnitt{Summary \textit{stackless} resumable function/lambda}
\begin{itemize}
    \item   for each resumable function/lambda an activation frame is allocated
    \item   activation frame (local variables, control block) and stack are
            distinct
    \item   stack is shared between resumable functions/lambdas
    \item   asymmetric context switching -> tight coupling between caller and
            invoker
    \item   returning data from resumable function/lambda via \yield etc.
    \item   no intuitive passing data into body of resumable function/lambda
\end{itemize}

\uabschnitt{Summary \textit{stackful} execution context}
\begin{itemize}
    \item   each stackful execution context owns a stack (fixed-size or linked
            stack)
    \item   activation frame and stack (function frames) are the same; no
            distinction
    \item   stack is not shared between execution context's
    \item   symmetric context switching -> no coupling between caller and
            invoker (more efficient in the context of cooperative multitasking)
    \item   passing data via lambda captures
\end{itemize}
