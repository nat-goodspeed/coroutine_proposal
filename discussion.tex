\abschnitt{Discussion}
\uabschnitt{Calling subroutines}
The advantage of stackless coroutines is that they reuse the same linear
processor stack for stack frames for called subroutines. The advantage of
stackful context switching is that it permits {\bfseries suspending from
nested calls}.\\
\newline
If a resumable function calls a traditional function (rather than another
resumable function), then the activation record belonging to the traditional
function is allocated on the processor stack (so it is called a stack frame).
As a consequence, stack frames of called functions must be removed from the
processor stack before the resumable function yields back to its caller.\\
In other words: the calling convention of the {\bfseries ABI dictates} that,
when the resumable function returns (suspends), the stack pointer must contain
the same address as before the resumable function was entered.\\
Hence a yield from nested call is {\bfseries not permitted} -- unless every
called function down to the yield point is also a resumable function.\\
The benefit of stackless coroutines consists in reusing the processor stack
for called subroutines: no separate stack memory need be allocated.\\
\newline
Of course even a stackless resumable function might fail if its called
functions exhaust the available stack.\\
\newline
In stackful context switching, each execution context owns a
distinct side stack which is assigned to the stack pointer (thus the stack
pointer must be exchanged during each context switch).\\
All activation records (stack frames) of subroutines are placed on the side
stack. Hence each stackful execution context requires enough memory to hold
the stack frames of the longest call chain of subroutines. Therefore, to
support calling subroutines, stackful context switching has a higher memory
footprint than resumable functions.\\
On the other hand, it is beneficial to use side stacks because the stack
frames of active subroutines remain intact while the execution context is
suspended. This is the reason why stackful context switching permits
{\bfseries yielding from nested calls}.

\uabschnitt{Why symmetric transfer of control matters}
As a building block for user-mode threads, symmetric control transfer is more
efficient than the asymmetric mechanism.\\
\newline
Asymmetric coroutines (as implied with keywords like \resumable, \await or
\yield) provide two operations for context switching. The caller and the
coroutine are coupled, that is, such a coroutine can only jump back to the code
that most recently resumed it.
\graphc{asymm}
For {\bfseries N} asymmetric coroutines, {\bfseries 2N} context switches are
required. This is sufficient in the case of generators, but in the context of
cooperative multitasking it is inefficient.\\
\newline
The proposed stackful execution context (\ectx) provides only one operation to
resume/suspend the context (\op). Control is directly transferred from one
execution context to another (symmetric control transfer) - no jump back to
the caller. In addition to supporting generators, this enables an efficient
implementation of cooperative multitasking: {\bfseries no additional context
switch} back to caller, direct context switch to next task.\\
The next execution context must be explicitly specified.\\
\newline
\graphc{symm}
Resuming {\bfseries N} instances of \ectx takes {\bfseries N+1} context
switches.

\uabschnitt{First class object}
Symmetric control transfer requires that the context is represented by a
{\bfseries first class object} (context that has to be resumed next must be
selectable).

\uabschnitt{Toplevel functions: main() and thread entry-functions}
\main and \emph{thread-entry functions} are represented by \ectx too.
\cppf{simple}
Instance \cpp{l1} represents the execution context of \main.


\uabschnitt{Stack strategies}
For stackful coroutines two strategies are typical: a contiguous, fixed-size
stack (as used by threads), or a linked stack (grows on demand).\\
The advantage of a fixed-size stack is the fast allocation/deallocation of 
activation records. A disadvantage is that the required stacksize must
be guessed.\\
The benefit of using a linked stack is that only the initial size of the stack
is required. The stack itself grows on demand, by means of an overflow
handler. The performance penalty is low. The disadvantage is that code
executed inside a stackful coroutine must be compiled for this stack model. In
the case of GCC's split stacks, special compiler/linker flags must be
specified - no changes to source code are required.\\
When calling a library function not compiled for linked stacks (expecting a
traditional contiguous stack), GCC's implementation uses link-time code
generation to change the instructions in the caller. The effect is that a
reasonably large contiguous stack chunk is temporarily linked in to handle the
deepest expected chain of traditional function stack frames (see GCC's
documentation\cite{gccsplit}).
