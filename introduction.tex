\abschnitt{Introduction}
Traditionally C++ code is compiled for a linear stack. That means that the
activation records are allocated in strict \emph{last-in-first-out} order. This
stack model allocates activation records on function call/return by
incrementing/decrementing the stack pointer.\\
Coroutines enable a more advanced control flow. That requires that a coroutine
outlives the context in which it was created (the activation record of a
suspended coroutine {\bfseries must not} be {\bfseries removed})!
Introducing fixed activation records into the middle of a linear stack
essentially makes normal stack operations unworkable.\\
Traditional stack management is inadequate for coroutines.\\
\newline
N4397\cite{N4397} describes the \emph{first-class} construct \ectx, representing
an execution state. A program implicitly contains at least one execution context. As
explained in N4397, \ectx can be used to implement stackful coroutines as
proposed in N3985\cite{N3985} or might be the building block of \emph{oneshot
shift/reset}
operators\footnote{\href{https://www.gnu.org/software/guile/manual/html_node/Shift-and-Reset.html}
{Shift, Reset, and All That}}.\\
In the remaining chapters the proposal describes under which constraints \ectx
can be used for stackless and stackful context switching.
