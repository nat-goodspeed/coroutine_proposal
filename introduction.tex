\abschnitt{Introduction}

This proposal suggests to add two first-class, semi-asymmetric continuations to
C++ standard library: \pullcoro and \pushcoro.\\
\newline
In computer science routines are defined as a sequence of operations. The
execution of routines form a parent-child relationship and the child terminates
always before the parent. Coroutines (the term was introduced by Melvin
Conway\cite{Conway1963}) are a generalization of routines (Donald
Knuth\cite{Knuth1997}). The principal difference between coroutines and routines
is that a coroutine enables explicit suspend and resume of their progress via
additional operations by preserving execution state and thus provide an
{\bf enhanced control flow} (maintaining the execution context).\\

\paragraph*{characteristics:}
Characteristics\cite{Moura2009} of a coroutine are:
\begin{itemize}
    \item values of local data persist between successive calls (context
          switches)
    \item execution is suspended as control leaves coroutine and resumed at
          certain time later
    \item symmetric or asymmetric control-transfer mechanism
    \item first-class object (can be passed as argument, returned by procedures,
          stored in a data structure to be used later or freely manipulated by
          the developer)
    \item stackful or stackless
\end{itemize}

Several programming languages adopted particular features (C\# yield, Python
generators, ...).
\begin{table}[h]
    \centering
    \begin{tabular}{ l l l l l l }
        \midrule
        BCPL    &   Erlang  &   Go      &   Lua         &   PHP     &   Ruby\\
        \midrule
        C\#     &   F\#     &   Haskell &   Modula-2    &   Prolog  &   Sather\\
        \midrule
        D       &   Factor  &   Icon    &   Perl        &   Python  &   Scheme\\
        \midrule
    \end{tabular}
    \caption{some programming languages with native support of coroutines
        \cite{wikipedia}}
\end{table}
\newline
Coroutines are useful in simulation, artificial intelligence, concurrent
programming, text processing and data manipulation\cite{Moura2009}, supporting
the implementation of components such as cooperative tasks (fibers), iterators,
generators, infinite lists, pipes etc.

\paragraph*{execution-transfer mechanism:}
Two categories of coroutines exist: symmetric and asymmetric coroutines.\\
\newline
A symmetric coroutine transfers the execution control only via one operation.
The target coroutine must be explicitly specified in the transfer operation.\\
\newline
Asymmetric coroutines provide two transfer operations:
the \textit{suspend}-operation returns to the invoker by preserving the
execution context  and the \textit{resume}-operation restores the execution
context, e.g. re-enters the coroutine at the same point as it was suspended
before.
\imgc{sequence.pdf}
Both concepts are equivalent and a general-purpose coroutine can provide either
symmetric or asymmetric coroutines\cite{Moura2009}.

\paragraph*{stackfulness:}
In contrast to a stackless coroutine a stackful coroutine allows to suspend
from nested stackframes. The execution resumes at exact the same point in the
code as it was suspended before.\\
Stackless coroutines limits the application for general-purpose multitasking
\cite{Moura2009} (for instance stackless coroutines can not be used to suspend
the current task if blocking operations are invoked from within a library).

\paragraph*{first-class continuation:}
A first-class continuations can be passed as argument, returned by
function and stored in a data structure to be used later.\\
In some implementations (for instance C\# \textit{yield}) the coroutine can not
directly accessed or directly manipulated.\\
\newline
Without stackfulness and first-class semantic some useful execution control
flows cannot be supported (for instance cooperative multitasking or
checkpointing).
