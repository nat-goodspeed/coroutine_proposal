\abschnitt{Introduction}
Traditionally C++ code is compiled for a linear stack, that means that the
activation records are allocated in strict \emph{last-in-first-out}-order. This
stack model allocates activation records on function call/return by
incrementing/decrementing the stack pointer.\\
Coroutines enable a more advanced control flow. That requires a coroutine
outlives the context they was created in. , but a linear stack prevents this.\\
But for a suspended coroutine the activation record {\bfseries must not} be
{\bfseries removed}! Additionaly, if the activation record of a suspended
coroutine remains on the linear processor stack and other code uses the stack
in the meanwhile, then all stack frames allocated after suspending the coroutine
might be corrupted, if the coroutine is resumed.\\
Traditional stack management is inadequate for coroutines.
