\abschnitt{Introduction}
In the past multiple proposals were published to support coroutines in C++.\\
It is astonishing that each proposal distinguishes coroutines into \emph{two}
kinds: \emph{stackless} and \emph{stackful}.\\
\newline
Traditionally C++ code is run on a linear stack, i.e. the activation
records are allocated in strict \emph{last-in-first-out} order. This stack
model allocates activation records on function call/return by
incrementing/decrementing the stack pointer.\\
But in the context of coroutines, that is, switching between different execution
contexts, a linear stack introduces problems. Calling a function creates an
activation record on the stack which is removed if the function returns.
But for a suspended coroutine the activation record {\bfseries must not} be
{\bfseries removed}!
\newline
Consider the following scenario:
\begin{itemize}
    \item Assume the processor stack is built in descending order: that is, a
          PUSH instruction decrements the stack pointer register. Call the
          stack pointer's initial value SP0.
    \item Function \main enters coroutine \cpp{C()}.
    \item \cpp{C()}'s prolog allocates a stack frame of
          size \cpp{sizeof(C::frame)} by decrementing SP. SP is now at SP1 =
          (SP0 - \cpp{sizeof(C::frame)}).
    \item \cpp{C()} suspends, returning control to \main. \main must find its
          stack frame at SP = SP0.
    \item \main now calls function \cpp{F()}.
    \item \cpp{F()}'s prolog allocates a stack frame of
          size \cpp{sizeof(F::frame)} by decrementing SP. SP is now at SP2 =
          (SP0 - \cpp{sizeof(F::frame)}).
    \item Unless either \cpp{(sizeof(C::frame) == 0)}
          or \cpp{(sizeof(F::frame) == 0)}, any data written by \cpp{F()} to
          its own stack frame will necessarily overwrite any data saved
          by \cpp{C()} in its stack frame.
    \item \cpp{F()} returns. SP is back to SP0.
    \item \main resumes \cpp{C()}. SP is set to SP1.
    \item \cpp{C()} attempts to access data in its stack frame -- which has
          been overwritten by \cpp{F()}. We are now in the realm of Undefined
          Behavior.
\end{itemize}

In order to prevent stack corruption, a stackless coroutine uses a heap-allocated
activation record (N4134\cite{N4134}), while stackful coroutines use a side
stack (N3985\cite{N3985}).\\
Since an N4134 stackless resumable function uses \emph{suspend by return},
when it suspends, the stack pointer is restored to its position before the
resumable function was called. While executing, a resumable function can
consume additional space in the linear processor stack; it can call
traditional functions. But they must all return before the resumable function
suspends. If resumable function A() calls function B(), and B() wishes to
suspend, B() must also be a resumable function. Thus the
term \emph{stackless}: a suspended resumable function leaves no activation
record on the linear processor stack. A single linear stack can be reused by
an arbitrary number of suspended resumable functions.\\
Using a side stack permits \emph{stackful} coroutines to use \emph{suspend by
call}. An arbitrary number of ordinary stack frames can be left on the side
stack for a suspended coroutine context; arbitrary stack frames can be pushed
or popped on the currently-active stack, independently of any suspended stack.\\
This is the fundamental difference between stackless and stackful coroutines.\\
\newline
Traditional stack management -- a single linear stack per thread -- is
inadequate for coroutines.
