\abschnitt{Motivation}
\uabschnitt{Content}
This paper proposes \ectx that serves as the crucial foundation on which a
number of valuable higher-level facilities can be built using pure C++.\\
\newline
\ectx itself is, however, impossible to code in portable C++. Having it provided
with the runtime library will be enabling technology for both third-party
libraries and user applications.\\

\uabschnitt{Why Bother?}
Given P0057 resumable functions, why should we even consider a completely
different mechanism for suspending and resuming a function? Isn't this
completely redundant with that?\\
\newline
The answer is ``no,'' for several different reasons.\footnote{The authors are
indebted to Christopher Kohlhoff for his excellent summary in
N4453,\cite{N4453} sections 4.1 and 4.2.}

\begin{itemize}
    \item With resumable functions, the markup to support suspension
          propagates virally through a code base. Resumable functions suspend
          by returning. Every caller must therefore distinguish between
          ``callee returned value'' and ``callee suspended, you must also
          suspend.'' The new \cpp{co\_await} keyword makes that distinction,
          suspending the containing function until the callee produces a
          value. Every call to a resumable function must itself
          use \cpp{co\_await} -- which then imposes the same requirement
          on \emph{its} caller, and so on.
    \item Forgetting to annotate a call to a resumable function
          with \cpp{co\_await} (perhaps because the caller is unaware of its
          nature, perhaps because the callee changed its nature) can result in
          subtle timing bugs. Suppose function \cpp{f()} returns
          a \cpp{std::future<void>}. The statement \cpp{co\_await f();} suspends
          the containing function until the \cpp{future} returned by \cpp{f()}
          is fulfilled. Coding simply \cpp{f();} merely \emph{discards} that
          future. Both forms are perfectly legal; both are likely to survive
          desk-checking and code review. It's even worse
          if \cpp{f()} \emph{usually} fulfills its \cpp{future} by the time it
          returns, only \emph{occasionally} taking longer. That way the error
          will survive most QA as well, and escape into production.
    \item The need to \cpp{co\_await} called functions means that we must evolve a
          whole new family of \cpp{co\_await}-aware STL
          algorithms.\footnote{N4453\cite{N4453} section 4.2 explains this
          more fully.}
    \item Even if one is willing to accept the viral \cpp{co\_await} markup of
          resumable functions, using normal encapsulation to manage layers of
          abstraction could become expensive in runtime. Every entry to a
          resumable function requires a heap allocation, freed on return. By
          contrast, once you have allocated a side stack, function call and
          return on that side stack is just as efficient as function call and
          return on the original application stack. It's a classic time/space
          tradeoff. (If you determine to address that issue by drastically
          pruning local variables from your resumable functions, note that the
          same tactic could allow you to use a far smaller side stack -- which
          would still be faster for nontrivial call chains.)
\end{itemize}
