\abschnitt{A. \textit{jump}-operation for SYSV ABI on x86\_64}\label{appendix}

The assembler code (from \boostcontext) shows what the \textit{jump}-operation
might look like for SYSV ABI on x86\_64.
\asmf{jump.S}

Register \asm{rdi} contains a reference to the \cblock of the current execution
context \textit{X} and register \asm{rsi} points to the \cblock of the target execution
context \textit{Y} to be resumed.\\
\newline
Register \asm{rdi} contains a reference to the \cblock of the current execution
In lines 20-25 the contents of the current non-volatile registers are stored in the
\cblock of \textit{X}.\\
\newline
Register \asm{rdi} contains a reference to the \cblock of the current execution
Line 29 calculates the stack-pointer writes it to the \cblock in line 30.\\
Lines 31 and 32 do the same for the return address (will be assigned to the
instruction-pointer \asm{rip}).\\
\newline
Register \asm{rdi} contains a reference to the \cblock of the current execution
The next block (lines 35-40) restores the contents of non-volatile registers for
the execution context \textit{Y}.\\
\newline
Register \asm{rdi} contains a reference to the \cblock of the current execution
The stack-pointer is restored in line 43.\\
Line 44 moves the address of the instruction which should be executed in
\textit{Y} to register \asm{rcx}.\\
\newline
Register \asm{rdi} contains a reference to the \cblock of the current execution
Lines 47 and 48 allow to transfer data (as return value in \textit{Y}) between
context jumps.\\
\newline
Register \asm{rdi} contains a reference to the \cblock of the current execution
The next line transfers execution control (\textit{branch-and-link}) to
\textit{Y} by executing the instruction to which \asm{rcx} points.
