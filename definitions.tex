\abschnitt{Definitions}

\uabschnitt{execution context}
environment where program logic is evaluated in (CPU registers, stack).

\uabschnitt{control block}
holds a set of registers (callee-saved registers, instruction pointer, stack
pointer) describing the execution context.

\uabschnitt{coroutine}
enables explicit suspend and resume of its progress via additional operations by
preserving execution state and thus provides an enhanced control flow.
Coroutines have following characteristics\cite{N3985}:
\begin{itemize}
    \item values of local data persist between successive context switches
    \item execution is suspended as control leaves coroutine and resumed at
          certain time later
    \item symmetric or asymmetric control transfer-mechanism
    \item first-class object or compiler-internal structure
    \item stackless or stackful
\end{itemize}

\uabschnitt{asymmetric coroutine}
provides two distingt operations for the context switch - one operation to
resume and one operation to suspend the coroutine.\\
An asymmetric coroutine is tightly coupled with its caller, e.g. suspending the
coroutine transferes the execution control back to the point in the code were it
was called. The asymmetric control transfer-mechanism is usually used in the
context of generators.

\uabschnitt{symmetric coroutine}
only one operation to resume/suspend the context is available.\\
A symmetric coroutine does not know its caller, e.g. the execution control can
be transferred to any other symmetric coroutine (must be explicitly specified).
The symmetric control transfer-mechanism is usually used to implement user-mode
threads.

\uabschnitt{fiber/user-mode thread}
execute tasks in an cooperative multitasking environment involving a
scheduler. Coroutines and fibers are distinct (N4024\cite{N4024}).

\uabschnitt{resumable function}
N4134\cite{N4134} describes resumable functions as an efficient language
supported mechanism for stackless context switching introducing two new keywords
- \await and \yield. Resumable functions are equivalent to asymmetric
coroutines.\\
Characteristics of resumable functions:
\begin{itemize}
    \item stackless
    \item allocates memory (activation frame) for the body of the resumable
          function to store local data, registers etc.
    \item thight coupling between caller and resumable function (asymmetric
          control transfer mechanism)
    \item implicit \textit{return}-statement\cite{N4134} (code transformation)
\end{itemize}

\uabschnitt{resumable lambda}
N4244\cite{N4244} describes resumable lambdas as stackless coroutines -
introducing new keywords\\
\resumable, flavours of \yield and \rlthis. Stackless resumable lambdas are
equivalent to asymmetric coroutines.\\
Characteristics of resumable lambdas:
\begin{itemize}
    \item stackless
    \item body of a lambda is available to the compiler at the point
          where the lambda type is defined
    \item compiler can analyse the lambda body to determine what stack
          variables need to be accommodated
    \item space can be made available for the stack variables in exactly the
          same way as is already done for the capture set
    \item thight coupling between caller and resumable lambda (asymmetric
          control transfer-mechanism)
\end{itemize}

\uabschnitt{application stack}
also known as call stack - is a chunk of memory assigned the stack pointer and
used to store information (for instance local variables) about the active
subroutines. The stack used by function \main grows on demand while the stack
assigned to a thread has a fixed size - usually 1MB (Windows) till 2MB (Linux),
but some platforms use smaller stacks (64KB on HP-UX/PA and 256KB on
HP-UX/Itanium).

\uabschnitt{side stack}
is a call stack that is used in the case of stackful context switching, e.g.
each execution context gets its own stack (assigned to stack pointer). Thus
stack frames of subroutines are allocated on the side stack, the application
stack remains unchanged.

\uabschnitt{activation frame}
is a chunk of memory used by resumable functions to store local (stack)
variables and the control block. Each resumable function is bound to its own
activation frame. The stack pointer remains unchanged, e.g. it still points to
the application stack. Thus stack frames of subroutines are allocated on the 
application stack.\\
Resumable lambdas re-use the storage of the capture set for the same purpose.

\uabschnitt{linked stack}
also known as \textit{split stack}\cite{gccsplit} or
\textit{segmented stack}\cite{llvmseg}, represents a stack with a non-contigous
address range.\\
Applications compiled with support for linked stacks can use (link against)
libraries not supporting linked stacks (see GCC's documentation\cite{gccsplit},
chapter 'Backward compatibility').
