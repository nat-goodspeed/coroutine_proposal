\abschnitt{Notes on wording \susup and \susdown}
The wording of \susup and \susdown, introduced in paper P0158\cite{P0158}, as
introduced to distinguish stackless (\emph{suspend-up}) and stackful (\susdown)
context switching.\\
\susup means \emph{removing the stack frame of the stackless context} while
\susdown corresponds to \emph{the frame of the context remains on the stack} on
suspention.\\
The autors believe that this wording is bit missleading, because \emph{up} and
\emph{down} imply that the stack grows from higher to lower addresses (which is
not true for some architectures).\\
\newline
Instead \susup and \susdown, the authros suggest to use the wording \susreturn for
stackless and \suscall for stackful context switching (described in
N4397\cite{N4397}).\\
\susreturn means \emph{remove the frame of the coroutien from the stack} as the
keyword \emph{return} already stands for in the C++ standard. \suscall does
suspend the current context by calling function\\
\cpp{std::execution\_context<>::operator()} while the frame remains on
context' stack.
