\abschnitt{Definitions}

\uabschnitt{Execution context}
environment in which program logic is evaluated (CPU registers, stack).

\uabschnitt{Control block}
holds a set of registers (callee-saved registers, instruction pointer, stack
pointer) describing the execution context.

\uabschnitt{Coroutine}
enables explicit suspend and resume of its progress via additional operations by
preserving execution state and thus provides an enhanced control flow.
Coroutines have following characteristics\cite{N3985}:
\begin{itemize}
    \item values of local data persist across context switches
    \item execution is suspended as control leaves coroutine and resumed at
          certain time later
    \item symmetric or asymmetric control transfer-mechanism
    \item first-class object or compiler-internal structure
    \item stackless or stackful
\end{itemize}

\uabschnitt{Asymmetric coroutine}
provides two distinct operations for the context switch - one operation to
resume and one operation to suspend the coroutine.\\
An asymmetric coroutine is tightly coupled with its caller, e.g. suspending the
coroutine transfers the execution control back to the point in the code where it
was called. The asymmetric control-transfer mechanism is usually used in the
context of generators.

\uabschnitt{Symmetric coroutine}
only one operation to resume/suspend the context is needed.\\
A symmetric coroutine does not know its caller; control can
be transferred to any other symmetric coroutine (which must be explicitly specified).
The symmetric control-transfer mechanism is usually used to implement
cooperative multitasking.

\uabschnitt{Fiber/user-mode thread}
execute tasks in a cooperative multitasking environment involving a scheduler.
Coroutines and fibers are distinct (N4024\cite{N4024}).

\uabschnitt{Resumable function}
N4134\cite{N4134} describes resumable functions as an efficient 
language-supported mechanism for stackless context switching introducing two new keywords
- \await and \yield. Resumable functions are equivalent to asymmetric
coroutines.\\
Characteristics of resumable functions:
\begin{itemize}
    \item stackless
    \item allocates memory (activation frame) for the body of the resumable
          function to store local data and control block
    \item tight coupling between caller and resumable function (asymmetric
          control transfer-mechanism)
    \item implicit \emph{return}-statement\cite{N4134} (code transformation)
\end{itemize}

\uabschnitt{Processor stack}
also known as call stack - is a chunk of memory into which the processor's
stack pointer is pointing. The processor stack might belong to:
\begin{itemize}
    \item an application's inital (or only) thread
    \item an explicitly-launched thread
    \item a sub-thread execution context (coroutine)
\end{itemize}

The stack is used to store data like local variables, content of CPU registers
(because of subroutine calls) etc. The processor stack assigned to function
\main grows on demand while the stack assigned to a thread has a fixed size -
usually 1MB (Windows) up to 2MB (Linux), but some platforms use smaller stacks
(64KB on HP-UX/PA and 256KB on HP-UX/Itanium).

\uabschnitt{Side stack}
is a call stack that is used in the case of stackful context switching, e.g.
each execution context gets its own stack (assigned to stack pointer). While
a given side stack is active, every other stack remains unchanged.

\uabschnitt{Activation frame}
is a chunk of memory used by resumable functions to store local (stack)
variables and the control block. Each resumable function is bound to its own
activation frame. The stack pointer remains unchanged, e.g. it still points to
the processor stack. Stack frames of subroutines are allocated on the 
processor stack.

\uabschnitt{Linked stack}
also known as \emph{split stack}\cite{gccsplit} or
\emph{segmented stack}\cite{llvmseg}, represents a stack that grows on demand
with a non-contigous address range.\\
Applications compiled with support for linked stacks can use (link against)
libraries not supporting linked stacks (see GCC's documentation\cite{gccsplit},
chapter 'Backward compatibility').

\uabschnitt{Parent pointer tree}
data structure in which each node has a pointer to its parent, but no pointer to
its children. Traversion from any node to its ancestors is possible but not to
any other node.\\
Used as call stack, the structure is called \emph{cactus stack}.
