\abschnitt{Definitions}

\uabschnitt{execution context}
environment where program logic is evaluated in.

\uabschnitt{control block}
holds a set of registers (callee-saved registers, instruction pointer, stack
pointer) describing the execution context.

\uabschnitt{coroutine}
enables explicit suspend and resume of its progress via additional operations by
preserving execution state and thus provides an enhanced control flow.\\
Coroutines have following characteristics\cite{N3985}:
\begin{itemize}
    \item values of local data persist between successive context switches
    \item execution is suspended as control leaves coroutine and resumed at
          certain time later
    \item symmetric or asymmetric control transfer-mechanism
    \item first-class object or compiler-internal structure
    \item stackless or stackful
\end{itemize}

\uabschnitt{asymmetric coroutine}
provides two distingt operations for contexts switch - one operation to
resume and one operation to suspend the coroutine.\\
A asymmetric coroutine is tightly coupled with its caller, e.g. suspending
the coroutine transferes the execution control back to its invoker (implicit
specified).\\
Usually used in the context of generators.

\uabschnitt{symmetric coroutine}
only one operation to resume/suspend the context is available.\\
A symmetric coroutine does not know its caller, e.g. the execution control can
passed to any other symmetric coroutine (must be explicitly specified).\\
Usually used to implement user-mode threads.

\uabschnitt{fiber/user-mode thread}
execute tasks in an cooperative multitasking environment involving a
scheduler. Coroutines and fibers are distinct (see N4024\cite{N4024}).

\uabschnitt{resumable function}
N4134\cite{N4134} describes resumable functions as an efficient language
supported mechanism for stackless coroutines introducing two new keywords -
\await and \yield. Resumable functions are equivalent to asymmetric coroutines.\\
Characteristics of resumable functions:
\begin{itemize}
    \item stackless
    \item allocates memory (activation frame) for the body of the resumable
          function to store local data, registers etc.
    \item thight coupling between caller and resumable function (asymmetric
          control transfer-mechanism)
    \item implicit \textit{return}-statement\cite{N4134} (code-transformation)
\end{itemize}

\uabschnitt{resumable lambdas}
N4244\cite{N4244} describes resumable lambdas as stackless coroutines -
introducing new keywords \resumable, flavours of \yield and \rlthis. Stackless
resumable lambdas are equivalent to asymmetric coroutines.\\
Characteristics of resumable lambdas:
\begin{itemize}
    \item stackless
    \item body of a lambda is available to the compiler at the point
          where the lambda type is defined
    \item compiler can analyse the lambda body to determine what stack
          variables need to be accommodated
    \item space can be made available for the stack variables in exactly the
          same way as is already done for the capture set
    \item thight coupling between caller and resumable lambda (asymmetric
          control transfer-mechanism)
\end{itemize}

\uabschnitt{linked stack}
also known as \textit{split stack}\cite{gccsplit} or
\textit{segmented stack}\cite{llvmseg}, represents a stack with a non-contigous
address-range.\\
Applications compiled with support for linked stacks can use (link against)
libraries not supporting linked stacks (see GCC's documentation\cite{gccsplit},
chapter 'Backward compatibility').
