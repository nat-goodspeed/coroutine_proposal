\abschnitt{Definitions}
This section gives an overview about the wording used in this proposal.

\uabschnitt{Execution context}
environment in which program logic is evaluated (CPU registers, stack).

\uabschnitt{Activation record}
also known as \emph{activation frame} (\emph{stack frame}
is a special activation record).\\
An \emph{activation record} is a data structure containing the state of a
particular function call. The following data are part of an activation record:
\begin{itemize}
    \item a set of registers (defined by the calling convention of the ABI
        \footnote{Application Binary Interface}, e.g. the return address)
    \item local variables
    \item parameters to the function
    \item eventually return value
\end{itemize}

\uabschnitt{Stack frame}
an activation record allocated on the processor stack; stack frames are
allocated in strict last-in-first-out order - placed by
incrementing/decrementing the stack pointer register on function call/return.

\uabschnitt{Heap-allocated activation record}
an activation record allocated in heap storage; every heap activation record
holds a pointer to its parent activation record (parent pointer tree).

\uabschnitt{Processor stack}
is a chunk of memory into which the processor's stack pointer register is
pointing. The processor stack might belong to:
\begin{itemize}
    \item an application's initial (or only) thread
    \item an explicitly-launched thread
    \item a sub-thread execution context (e.g. a coroutine)
\end{itemize}

\uabschnitt{Application stack}
The processor stack assigned to function \main is generally a \emph{linear
stack}. Often, the memory management system is used to detect stack
overflow and allocate more memory. This stack is ideally placed well away from
any other data in the process's address space so it can be extended as needed.

\uabschnitt{Thread stack}
The processor stack assigned to an explicitly-launched thread is generally
a \emph{linear stack}. In a 32-bit (or smaller) address space, the operating
system cannot, in general, guarantee that such a stack can be arbitrarily
extended on overflow: the range of available addresses is constrained by the
next adjacent allocated memory block. Instead, stacks with a fixed size are
used - usually 1MB (Windows) up to 2MB (Linux), but some platforms use smaller
stacks (64KB on HP-UX/PA and 256KB on HP-UX/Itanium).

\uabschnitt{Side stack}
is a processor stack that is used for \emph{stackful} context switching. Each
execution context gets its own stack. The stack belonging to an inactive
context remains unchanged, while function calls and returns push and pop stack
frames on the currently-active stack.

A side stack is generally allocated by the running process (library code)
rather than by the operating system. It might either be a \emph{linear stack}
or a \emph{linked stack}.

\uabschnitt{Linear stack}
is a contiguous memory area into which the processor's stack pointer register
points. On a processor with a stack pointer register, this is the traditional
stack model for C++. Allocation and deallocation of activation records
(stack frames in this context) is performed by incrementing/decrementing the
stack pointer.

\uabschnitt{Linked stack}
also known as \emph{split stack}\cite{gccsplit} or \emph{segmented
stack}\cite{llvmseg}. This is a linked list of contiguous memory blocks of
intermediate size, each of which might hold several function stack frames. The
processor's stack pointer points into the head block on this list; normally a
new stack frame is allocated by adjusting the stack pointer, as usual. When
compiler-generated code detects near overflow, it links on a new head block.
Thus, the effective stack space can grow arbitrarily without requiring a
single contiguous address range.

This mechanism has a low performance penalty - typically adding around 5-10\%
overhead to the instruction sequence implementing a function call. Only an
application consisting solely of empty functions would see that impact
overall. Small functions would see the most overhead -- but modern C++
compilers work really hard to inline small functions, and of course the
overhead vanishes entirely for inline functions.

Applications compiled with support for linked stacks can use (link against)
libraries not supporting linked stacks. (See GCC's documentation\cite{gccsplit},
chapter 'Backward compatibility'.)

\uabschnitt{Non-contiguous stack}
is a \emph{parent pointer tree}-structure used to store heap-allocated
activation records. It is a stack with branches; an activation record can
outlive the context in which it was created. With this mechanism, the
processor's stack pointer register does not point to (or into) any of these
activation records. Compare to \emph{Linked stack}.

A heap activation record is allocated on entry to a function and freed on
return, as with N4134\cite{N4134}.

This structure is also called \emph{cactus stack}\cite{cactusstack}
or \emph{spaghetti stack}.

\uabschnitt{Parent pointer tree}
data structure in which each node has a pointer to its parent, but no pointer to
its children. Traversal from any node to its ancestors is possible but not to
any other node.

\uabschnitt{Coroutine}
function that enables explicit suspend and resume of its progress by
preserving execution state (\emph{activation record}), thus providing an
enhanced control flow. Coroutines have the following characteristics\cite{N3985}:
\begin{itemize}
    \item values of local data persist across context switches
    \item execution is suspended as control leaves coroutine and resumed at
          certain time later
    \item symmetric or asymmetric control-transfer mechanism
    \item first-class object or compiler-internal structure
    \item stackless or stackful
\end{itemize}

\uabschnitt{Toplevel context function}
is a function executed as a coroutine (stackless or stackful). This term is
particularly useful when discussing library-supported context switching, as
the execution context must be constructed explicitly and passed a function.

\uabschnitt{Asymmetric coroutine}
provides two distinct operations for the context switch - one operation to
resume and one operation to suspend the coroutine.\\
An asymmetric coroutine is tightly coupled with its caller, i.e. suspending
the coroutine transfers execution control back to the point in the code
from which the coroutine was last resumed. The asymmetric control-transfer
mechanism is usually used in the context of generators.

\uabschnitt{Symmetric coroutine}
only one operation to resume/suspend the context is needed.\\
A symmetric coroutine does not know its caller; control can
be transferred to any other symmetric coroutine (which must be explicitly specified).
The symmetric control-transfer mechanism is usually used to implement
cooperative multitasking.

\uabschnitt{Fiber/user-mode thread}
execute tasks in a cooperative multitasking environment involving a scheduler.
Coroutines and fibers are distinct (N4024\cite{N4024}).

Code running in a user-mode thread (or ``fiber'') suspends by passing control
to the scheduler, which selects the next fiber to resume. Such code is not
inherently coupled to the code that launched the fiber (as with an asymmetric
coroutine); nor must it explicitly select the next fiber (as with a symmetric
coroutine).

\uabschnitt{Resumable function}
N4134\cite{N4134} describes resumable functions as an efficient
language-supported mechanism for stackless context switching introducing two
new keywords - \await and \yield. A resumable function is a kind of asymmetric
coroutine: suspending passes control back to its caller.\\
Characteristics of resumable functions:
\begin{itemize}
    \item stackless
    \item uses heap allocated activation records (referred as activation frames
          in N4134)
    \item tight coupling between caller and resumable function (asymmetric
          control-transfer mechanism)
    \item implicit \emph{return}-statement\cite{N4134} (code transformation)
\end{itemize}

\uabschnitt{Suspend by return}
The terms \emph{stackless} and \emph{stackful} can become confusing as
different implementations are discussed. An N4134\cite{N4134} ``stackless''
resumable function might consume a bit of the current processor stack on
entry, cleaning it off on suspension. A \emph{cactus stack} coroutine
implementation might provide ``stackful'' semantics while completely avoiding
the processor stack.

\emph{Suspend by return} describes a mechanism such as described in N4134\cite{N4134} and
N4244\cite{N4244}, in which suspending is implemented by returning from the
function body.

\uabschnitt{Suspend by call}
describes a mechanism such as described in N3985\cite{N3985}, in which
suspending is implemented by calling some other (library) function.
