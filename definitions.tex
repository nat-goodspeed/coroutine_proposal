\abschnitt{Notes on \susup and \susdown terminology}
The terms \susup and \susdown were introduced in paper N4232\cite{N4232} and
carried forward in P0158\cite{P0158} to distinguish stackless
(\susup) and stackful (\susdown) context switching.

These terms rely on a particular visualization of the C++ function call
operation in which calling a function passes control ``downwards,'' whereas
returning from a function passes control ``upwards.''

The authors recommend the terms \susreturn instead of \susup, and \suscall
instead of \susdown. The recommended terminology directly references the
underlying C++ operations, without requiring a particular visualization.

\susreturn (\susup, or ``stackless'' context switching) is based on returning
control from a called function to its caller, along with some indication as to
whether the called function has completed and is returning a result or is
merely suspending and expects to be called again. The called function's body
is coded in such a way that -- if it suspended -- calling it again will direct
control to the point from which it last returned.

This describes both P0057\cite{P0057} resumable functions and earlier
technologies such as Boost.Asio coroutines\cite{basiocoro}.

\suscall (\susdown, or ``stackful'' context switching) is based on calling a
function which, transparently to its caller, switches to some other logical
chain of function activation records. (This may or may not be a contiguous
stack area. The processor's stack pointer register, if any, may or may not
be involved.)

This describes N4397\cite{N4397} coroutines as well as
Boost.Context\cite{bcontext}, Boost.Coroutine2\cite{bcoroutine2} and
Boost.Fiber\cite{bfiber}.

\cpp{std::execution\_context<>::operator()} requires \suscall semantics.
