\abschnitt{B. \emph{std::execution\_context<>} as Thread of Execution (ToE)}\label{appendixb}
As described in P0073\cite{P0073}, a \ectx represents a ToE; each \ectx has
its own execution path (sequence of instructions).\footnote{In the terminology
used by P0072\cite{P0072}, this ToE runs on a ``weakly parallel'' execution
agent.} An operating system thread
(\cpp{std::thread}) consists of at least one \ectx representing the
main-stack/thread-stack.\\
Multiple \ectx instances might interact in following manner:
\begin{itemize}
\item an {\bfseries asymmetric coroutine} involves two \ectx s simply passing
      control back and forth to each other
    \begin{itemize}
        \item the two are strongly coupled; the callee \ectx can only resume
            its caller
        \item they directly exchange data (in general this could be
            bidirectional; a generator restricts data flow to a single
            direction)
    \end{itemize}
\item each {\bfseries fiber} runs on a \ectx of its own
    \begin{itemize}
        \item fibers are loosely coupled: a scheduler selects the next ready fiber
        \item no direct data transfer
    \end{itemize}
\item {\bfseries shift/reset operators} involve a few interacting \ectx s
    \begin{itemize}
        \item coupled
        \item data exchanged
        \item interleaved transfer of execution
    \end{itemize}
\end{itemize}
