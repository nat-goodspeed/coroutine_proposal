\setlength{\parindent}{0pt} 
\renewcommand\sfdefault{phv}

\makeatletter
    \renewcommand*\l@subsection{\@dottedtocline{2}{0em}{2.3em}}
    \renewcommand*\l@subsection{\@dottedtocline{3}{0em}{3.2em}}
    \renewcommand{\tableofcontents}{%
        \@starttoc{toc}
    }
\makeatother

%\renewcommand{\thesubsection}{\Roman{subsection}}

\newcommand{\pdfimg}[1]{\pdfximage{pics/#1}\pdfrefximage\pdflastximage}
\newcommand{\img}[1]{\mbox{\pdfimg{#1}}}
\newcommand{\imgc}[1]{\begin{center}\img{#1}\end{center}}

\lstdefinelanguage
   [x64]{Assembler}     % add a "x64" dialect of Assembler
   [x86masm]{Assembler} % based on the "x86masm" dialect
   % with these extra keywords:
   {morekeywords={rax,rdx,rcx,rbx,rsi,rdi,rsp,rbp,rip,r8,r9,r10,r11,r12,r13,r14,r15,popq,pushq,movq,subq,addq,xorq,leaq}}
\lstset{
        language=[x64]Assembler,
        numbers=none,
        numberstyle=\tiny,
        numberblanklines=false,
        stepnumber=1,
        numbersep=10pt
}

\newcommand{\cpp}[1]{
    \lstinline[
        language=C++,
        basicstyle=\ttfamily\color{black},
        keywordstyle=\color{blue},
        commentstyle=\color{green},
        stringstyle=\color{red},
    ] {#1}
}
\newcommand{\cppf}[1]{
    \lstinputlisting[
        language=C++,
        basicstyle=\ttfamily\color{black},
        keywordstyle=\color{blue},
        commentstyle=\color{red},
        stringstyle=\color{green},
    ] {code/#1}
}

\newcommand{\asm}[1]{
    \lstinline[
        basicstyle=\ttfamily\color{black},
        keywordstyle=\color{blue},
        commentstyle=\color{red},
        stringstyle=\color{green}
    ] {#1}
}
\newcommand{\asmf}[1]{
    \lstinputlisting[
        numbers=left,
        basicstyle=\ttfamily\color{black},
        keywordstyle=\color{blue},
        commentstyle=\color{red},
        stringstyle=\color{green}
    ] {code/#1}
}

\newcommand{\acoro}{\cpp{std::asymmetric_coroutine<T>}\xspace}
\newcommand{\asyncread}{\cpp{async_read}\xspace}
\newcommand{\asyncwrite}{\cpp{async_write}\xspace}
\newcommand{\bgin}{\cpp{std::begin()}\xspace}
\newcommand{\callcoro}{\cpp{std::symmetric_coroutine<T>::call_type}\xspace}
\newcommand{\callcorobool}{\cpp{std::symmetric_coroutine<T>::call_type::operator bool()}\xspace}
\newcommand{\callcoroop}{\cpp{std::symmetric_coroutine<T>::call_type::operator()()}\xspace}
\newcommand{\ed}{\cpp{std::end()}\xspace}
\newcommand{\get}{\cpp{get()}\xspace}
\newcommand{\pullcorobool}{\cpp{std::asymmetric_coroutine<T>::pull_type::operator bool()}\xspace}
\newcommand{\pullcoro}{\cpp{std::asymmetric_coroutine<T>::pull_type}\xspace}
\newcommand{\pullcoroget}{\cpp{std::asymmetric_coroutine<T>::pull_type::get()}\xspace}
\newcommand{\pullcoroiterator}{\cpp{std::asymmetric_coroutine<T>::pull_type::iterator}\xspace}
\newcommand{\pullcoroop}{\cpp{std::asymmetric_coroutine<T>::pull_type::operator()()}\xspace}
\newcommand{\pushcorobool}{\cpp{std::asymmetric_coroutine<T>::push_type::operator bool()}\xspace}
\newcommand{\pushcoro}{\cpp{std::asymmetric_coroutine<T>::push_type}\xspace}
\newcommand{\pushcoroiterator}{\cpp{std::asymmetric_coroutine<T>::push_type::iterator}\xspace}
\newcommand{\pushcoroop}{\cpp{std::asymmetric_coroutine<T>::push_type::operator()(Arg&&)}\xspace}
\newcommand{\scoro}{\cpp{std::symmetric_coroutine<T>}\xspace}
\newcommand{\tuple}{\cpp{std::tuple<>}\xspace}
\newcommand{\yieldcoro}{\cpp{std::symmetric_coroutine<T>::yield_type}\xspace}
\newcommand{\yieldcoroop}{\cpp{std::symmetric_coroutine<T>::yield_type::operator()()}\xspace}
\newcommand{\yieldcoroget}{\cpp{std::symmetric_coroutine<T>::yield_type::get()}\xspace}

\newcommand{\asyncops}{\textit{asynchronous-operations}\xspace}
\newcommand{\asyncres}{\textit{asynchronous-result}\xspace}
\newcommand{\async}{\textit{async}\xspace}
\newcommand{\await}{\textit{await}\xspace}
\newcommand{\cblock}{\textit{control-block}\xspace}
\newcommand{\checkpointing}{\textit{checkpointing}\xspace}
\newcommand{\checkpoint}{\textit{checkpoint}\xspace}
\newcommand{\continuation}{\textit{continuation}\xspace}
\newcommand{\coopmultitasking}{\textit{cooperative multitasking}\xspace}
\newcommand{\corofunction}{\textit{coroutine-function}\xspace}
\newcommand{\csharp}{\textit{C\#}\xspace}
\newcommand{\escreccomps}{\textit{\escre~recursive computations}\xspace}
\newcommand{\escrecloops}{\textit{\escre~loops}\xspace}
\newcommand{\escreops}{\textit{\escre~operations}\xspace}
\newcommand{\escre}{\textit{escape-and-reenter}\xspace}
\newcommand{\resumfn}{\textit{resumable function}\xspace}
\newcommand{\stack}{\textit{stack}\xspace}

\newcommand{\ABI}{ABI \cite{abi}\xspace}
\newcommand{\awaitemu}{await\_emu \cite{awaite}\xspace}
\newcommand{\boostasio}{boost.asio \cite{asio155}\xspace}
\newcommand{\boostcontext}{boost.context \cite{context155}\xspace}
\newcommand{\boostcoroutine}{boost.coroutine \cite{coroutine155}\xspace}
\newcommand{\boostfiber}{boost.fiber \cite{fiber}\xspace}
\newcommand{\cv}{calling convention \cite{callingconvention}\xspace}
\newcommand{\channelnine}{Channel 9 - 'The Future of C++' \cite{awaite}\xspace}
\newcommand{\yieldcontext}{yield\_context \cite{yieldc}\xspace}

\newcommand{\abschnitt}[1]{
    \addcontentsline{toc}{subsection}{#1}
    \subsection*{#1}
}

\newcommand{\anhang}[1]{
    \addcontentsline{toc}{subsection}{#1}
    \subsection*{#1}
}
