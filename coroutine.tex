\pdfoutput=1
\pdfcompresslevel=9
\pdfinfo
{
	/Title (A proposal to add coroutines to the C++ standard library)
	/Author (Oliver Kowalke, Nat Goodspeed)
	/Keywords (C++ coroutine )
}

%//////////////////////////////////////////////////////////////////////////////

\documentclass[a4paper,10pt,DIV15]{scrartcl}

\usepackage{overcite}
\usepackage[british]{babel}
\usepackage[latin1]{inputenc}
\usepackage[T1]{fontenc}
\usepackage{booktabs}
\usepackage{fancyhdr}
\usepackage{listings}
\usepackage{rotating}
\usepackage{xcolor}
\usepackage{xspace}
\usepackage[colorlinks=true,urlcolor=blue]{hyperref}

%//////////////////////////////////////////////////////////////////////////////

\setlength{\parindent}{0pt} 
\renewcommand\sfdefault{phv}

\makeatletter
    \renewcommand*\l@subsection{\@dottedtocline{2}{0em}{2.3em}}
    \renewcommand*\l@subsection{\@dottedtocline{3}{0em}{3.2em}}
    \renewcommand{\tableofcontents}{%
        \@starttoc{toc}
    }
\makeatother

%\renewcommand{\thesubsection}{\Roman{subsection}}

\newcommand{\pdfimg}[1]{\pdfximage{pics/#1}\pdfrefximage\pdflastximage}
\newcommand{\img}[1]{\mbox{\pdfimg{#1}}}
\newcommand{\imgc}[1]{\begin{center}\img{#1}\end{center}}

\newcommand{\cpp}[1]{
    \lstinline[
        language=C++,
        basicstyle=\ttfamily\color{black},
        keywordstyle=\color{blue},
        commentstyle=\color{green},
        stringstyle=\color{red},
    ] {#1}
}
\newcommand{\cppf}[1]{
    \lstinputlisting[
        language=C++,
        basicstyle=\ttfamily\color{black},
        keywordstyle=\color{blue},
        commentstyle=\color{red},
        stringstyle=\color{green},
    ] {code/#1}
}

\newcommand{\await}{\textit{await}\xspace}
\newcommand{\cblock}{\textit{control-block}\xspace}
\newcommand{\coopmultitasking}{\textit{cooperative multitasking}\xspace}
\newcommand{\corofunction}{\textit{coroutine-function}\xspace}
\newcommand{\resumable}{\textit{resumable}\xspace}
\newcommand{\resumfn}{\textit{resumable function}\xspace}
\newcommand{\resumld}{\textit{resumable lambda}\xspace}
\newcommand{\sfull}{\textit{stackful}\xspace}
\newcommand{\sfcoro}{\textit{stackful} coroutine\xspace}
\newcommand{\sfcoros}{\textit{stackful} coroutines\xspace}
\newcommand{\sless}{\textit{stackless}\xspace}
\newcommand{\slcoro}{\textit{stackless} coroutine\xspace}
\newcommand{\slcoros}{\textit{stackless} coroutines\xspace}
\newcommand{\stack}{\textit{stack}\xspace}
\newcommand{\yield}{\textit{yield}\xspace}

\newcommand{\abschnitt}[1]{
    \addcontentsline{toc}{subsection}{#1}
    \subsection*{#1}
}

\newcommand{\uabschnitt}[1]{
    \addcontentsline{toc}{paragraph}{#1}
    \paragraph*{#1}
}


%//////////////////////////////////////////////////////////////////////////////

\begin{document}

\small
\begin{tabbing}
    Document number: \=  \\
    Date:            \> 2014-04-21 \\
    Project:         \> Programming Language C++, Library Evolution Group\\
    Reply-to:        \> Oliver Kowalke (oliver dot kowalke at gmail dot com)\\
                     \> Nat Goodspeed ( nat at lindenlab dot com)\\
\end{tabbing}

\section*{A proposal to add coroutines to the C++ standard library (Revision 1)}

%//////////////////////////////////////////////////////////////////////////////

\tableofcontents

%//////////////////////////////////////////////////////////////////////////////

\paragraph*{Changes in this revision}
This document supersedes N3708. A new kind of coroutines - \scoro - is introduced
and additional examples (like recursive SAX parsing) are added.\\
A section explains the benfits of using coroutines in the context of event-based
asynchronous model.

\abschnitt{Introduction}

This proposal proposes to add \coro to the C++ standard.\\
\newline
In computer science routines are defined as a sequence of operations. The
execution of routines form a parent-child relationship and the child terminates
always before the parent. Coroutines (the term was introduced by Melvin
Conway\cite{Conway1963}) are a generalization of routines (Donald Knuth\cite{Knuth1997}). The
principal difference between coroutines and routines is that a coroutine enables
explicit suspend and resume of their progress via additional operations by
preserving local state. Coroutines support the implementation of components such
as cooperative tasks, iterators, infinite lists and pipes.\\
\newline
\coro is a first class \continuation (can be passed as argument, returned
by procedure and stored in a data structure to be used later).\\
The context switch is cooperative, e.g. the programmer controls when a switch
will happen (the kernel is not involved in the switch between coroutines).\\
\newline
\imgc{sequence.pdf}

\abschnitt{Motivation}

This proposal refers to \boostcoroutine as reference implementation - providing
a test suite and examples (some are described in this section).\\
\newline
In order to support a broad range of execution control behaviour the coroutine
types of \scoro and \acoro can be used to \escrecloops, to \escreccomps and for
\coopmultitasking helping to solve problems in a much simpler and more elegant
way than with only a single flow of control.

\subsubsection*{event-driven model}
The event-driven model is a programming paradigm where the flow of a program is
determined by events. The events are generated by multiple independent sources
and an event-dispatcher, waiting on all external sources, triggers callback
functions (event-handlers) whenever one of those events is detected (event-loop).
The application is divided into event selection (detection) and event handling.
\imgc{event_model.pdf}

The resulting applications are highly scalable, flexible, have high
responsiveness and the components are loosely coupled. This makes the event-driven
model suitable for user interface applications, rule-based productions systems
or applications dealing with asynchronous I/O (for instance network servers).


\subsubsection*{event-based asynchronous paradigm}
A classic synchronous console program issues an I/O request (e.g. for user
input or filesystem data) and blocks until the request is complete.
\newline
In contrast, an asynchronous I/O function initiates the physical operation but
immediately returns to its caller, even though the operation is not yet
complete. A program written to leverage this functionality does not block: it
can proceed with other work (including other I/O requests in parallel) while
the original operation is still pending. When the operation completes, the
program is notified. Because asynchronous applications spend less overall time
waiting for operations, they can outperform synchronous programs.
\newline
Events are one of the paradigms for asynchronous execution, but
not all asynchronous systems use events.
Although asynchronous programming can be done using threads, they come with
their own costs:

\begin{itemize}
    \item hard to program (traps for the unwary)
    \item memory requirements are high
    \item large overhead with creation and maintenance of thread state
    \item expensive context switching between threads
\end{itemize}

The event-based asynchronous model avoids those issues:

\begin{itemize}
    \item simpler because of the single stream of instructions
    \item much less expensive context switches
\end{itemize}

The downside of this paradigm consists in a sub-optimal program
structure. A event-driven program is required to split its code into
multiple small callback functions, i.e. the code is organized in a sequence of
small steps that execute intermittently. An algorithm that would usually be
expressed as a hierarchy of functions and loops must be transformed into
callbacks. The complete state has to be stored into a data structure while the
control flow returns to the event-loop.\\
As a consequence, event-driven applications are often tedious and confusing to
write. Each callback introduces a new scope, error callback etc. The
sequential nature of the algorithm is split into multiple callstacks,
making the application hard to debug. Exception handlers are restricted to
local handlers: it is impossible to wrap a sequence of events into a single
try-catch block.
The use of local variables, while/for loops, recursions etc. together with the
event-loop is not possible. The code becomes less expressive.\\
\newline
In the past, code using asio's \asyncops was convoluted by
callback functions.
\cppf{oldasio.cpp}

In this example, a simple echo server, the logic is split into three member
functions - local state (such as data buffer) is moved to member variables.\\
\newline
\boostasio provides with its new \asyncres feature a new
framework combining event-driven model and coroutines, hiding the complexity
of event-driven programming and permitting the style of classic sequential code.
The application is not required to pass callback functions to asynchronous
operations and local state is kept as local variables. Therefore the code
is much easier to read and understand.
Proposal 'N3964: Library Foundations for Asynchronous Operations'\cite{n3964}
describes the usage of coroutines in the context of asynchronous operations.\\
\yieldcontext internally uses \boostcoroutine:
\cppf{coroasio.cpp}

In contrast to the previous example this one gives the impression of sequential
code and local data while using asynchronous operations \asyncread,
\asyncwrite). The algorithm is implemented in one function and error handling is
done by one try-catch block.

\subsubsection*{'same fringe' problem}
The advantages can be seen particularly clearly with the use of a recursive
function, such as traversal of trees.\\
If traversing two different trees in the same deterministic order produces the
same list of leaf nodes, then both trees have the same fringe even if the tree
structure is different.\\
\newline
The same fringe problem could be solved using coroutines by iterating over the
leaf nodes and comparing this sequence via \cpp{std::equal()}. The range of data
values is generated by function \cpp{traverse()} which recursively traverses the
tree and passes each node's data value to its \pushcoro.\\
\pushcoro suspends the recursive computation and transfers the data value to
the main execution context.\\
\pullcoroiterator, created from \pullcoro, steps over those data values and
delivers them to \cpp{std::equal()} for comparison. Each increment of \pullcoroiterator
resumes \cpp{traverse()}. Upon return from \cpp{iterator::operator++()}, either
a new data value is available, or tree traversal is finished (iterator is
invalidated).
\cppf{same_fringe.cpp}

\subsubsection*{\csharp \await}
\csharp contains the two keywords \async and \await. \async introduces a
control flow that involves awaiting asynchronous operations. The compiler
reorganizes the code into a continuation-passing style. \await wraps the rest
of the function after calling \await into a continuation if the asynchronous
operation has not yet completed.\\
The project \awaitemu uses \boostcoroutine for a proof-of-concept
demonstrating the implementation of a full emulation of \csharp \await as a
library extension. Because of stackful coroutines \await is \textbf{not limited}
by "one level" as in \csharp.\\
Evgeny Panasyuk describes the advantages of \boostcoroutine over \await at
\channelnine.
\cppf{await.cpp}

\abschnitt{Impact on the Standard}
This proposal is a library extension. It does not require changes to any
standard classes, functions or headers. It can be implemented in C++03 and C++11
and requires no core language changes.

\abschnitt{Design}
Class \ectx provides a {\bfseries small, basic API} on which to build {\bfseries
higher-level APIs} such as stackful coroutines (N3985\cite{N3985}) and user-mode
threads (such as Boost.Fiber\cite{bfiber}).


\uabschnitt{Suspend-by-call}
\ectxop preserves the CPU register set\footnote{defined by ABI's calling
convention}: the content of those registers is pushed at the end of the stack
of the current context (at the current stack-pointer). Then \op restores the
stack-pointer register stored in \cpp{*this} and pops the CPU register set
from the newly-restored stack.
Because the context state is preserved on the context's stack, a \ectx
instance need only store the stack-pointer register.


\uabschnitt{Call semantics}
When \ectxop is called, a new instance of \ectx is synthesized representing
the current state of the running context (e.g. the stack-pointer). This new
instance is passed to the resumed context. On initial entry, it is passed as
the first argument to the top-level function. On every subsequent resumption,
it is returned by the suspended \op call.

On completion of a successful context switch, the
\ectx instance on which \op was called is invalidated. The data member from
which the stack pointer was just restored is set to \cpp{nullptr}.

At most one instance of \ectx can represent a given execution context. The
currently-running execution context is not represented by \emph{any} \ectx
instance. Only when \op is called on some \emph{other} \ectx instance is the
state of the running execution context captured in a synthesized \ectx
instance.

As mentioned in the section below on stack
destruction, \cpp{\~execution\_context<>()} on a suspended (not terminated)
instance destroys the stack managed by that instance. Thus, the stack must be
managed by only one \ectx instance.\footnote{An earlier design used reference
counting, but that subverts the intended role of this facility as an extremely
fast substrate for higher-level libraries.}

Because of the symmetric context switching (only one operation transfers
control), the target execution context must be explicitly specified.

\uabschnitt{std::execution\_context<void>}
With \cpp{execution\_context<void>} no data will be transferred, only the
context switch is executed.
\cppf{passing_void}
\cpp{ctx1()} resumes \cpp{ctx1}, that is, control enters the lambda passed to
the constructor of \cpp{ctx1}. Argument \cpp{ctx2} represents the previous
context: the context that was suspended by the call to \cpp{ctx1()}. When the
lambda returns \cpp{ctx2}, context \cpp{ctx1} will be terminated while the
context represented by \cpp{ctx2} is resumed, hence control returns
from \cpp{ctx1()}.\\

\uabschnitt{Passing data}
When you construct a \ectx with template arguments other than \cpp{void}, the
function or lambda that initializes that instance must accept parameters\\
\cpp{(std::execution\_context<args...>, args...)}, where \cpp{args...} here
represents any list of arguments other than \cpp{void}.

The initial \ectx argument is synthesized by \op. All other arguments must be
passed explicitly to \op.

The first call to \op with those arguments populates the parameter list for
the newly-entered function or lambda.

That function or lambda switches context back to the original context by
calling the passed\\
\ectxop, passing appropriate arguments.

The \emph{original} context's call to \op returns
a \cpp{std::tuple<std::execution\_context<args...>, args...>}. The
returned \ectx is a synthesized instance representing the context that just
suspended. The rest of the \cpp{args...} are as passed by that context to \op.

So, for instance:
\cppf{passing_single}
The \cpp{ctx1(i)} call at (a) enters the lambda in context \cpp{ctx1} with
argument \cpp{j=1}, as shown by the output at (b). The
expression \cpp{ctx2(j+1)} at (c) resumes the original context (represented
within the lambda by \cpp{ctx2}) and transfers back an integer of \cpp{j+1}.
On return from \cpp{ctx1(i)}, the assignment at (d) sets \cpp{i} to \cpp{j+1},
or 2.

The assignment at (d) illustrates a recommended idiom: since the call to \op
at (a) has invalidated \cpp{ctx1}, it should be replaced by the
newly-synthesized \ectx instance returned by \op.

To continue the example:
\cppf{passing_single_continued}
The call to \cpp{ctx1(i)} at (e) (the \emph{updated} \cpp{ctx1}) resumes
the \cpp{ctx1} lambda, returning from the \cpp{ctx2()} call at (c) and
executing the assignment at (f). Here, too, we replace the \ectx
instance \cpp{ctx2} invalidated by the \op call at (c) with the new instance
returned by that same \op call. Moreover, we replace \cpp{j} with the value
passed by the call at (e).

Finally the lambda returns (the updated) \cpp{ctx2} at (g), terminating its
context.

Since the updated \cpp{ctx2} represents the context suspended by the call at
(e), control returns to the assignment at (h). Once again we replace the
invalidated \cpp{ctx1} with the one returned by \op.

However, since context \cpp{ctx1} has now terminated, the updated \cpp{ctx1}
is \emph{not-a-context}. Its \cpp{operator bool()} returns \cpp{false};
its \cpp{operator\!()} returns \cpp{true}.

This is important, since in that case the values of any remaining fields of
the returned \cpp{std::tuple} are indeterminate.

It may seem tricky to keep track of which \ectx instance is currently
``live,'' representing the state of the suspended context. Please bear in
mind that this facility is intended as a high-performance foundation for
higher-level libraries. It is not intended to be directly consumed by
applications.\\
\newline
We can extend the example to multiple arguments.
\cppf{passing_multiple}
\op accepts the parameters specified by \ectx's template parameters. It
returns a \cpp{std::tuple} of that \ectx specialization, prepended to those
types.


\uabschnitt{Top-level thread functions}
\main as well as the \emph{entry-function} of a thread can be represented by an
execution context. That \ectx instance is synthesized when the running context
suspends, and is passed into the newly-resumed context.


\uabschnitt{\ectx and std::thread}
The context represented by a \ectx instance is necessarily suspended. It is
valid to resume a \ectx instance on any thread -- \emph{except} that you must
not attempt to resume a \ectx instance representing \main, or
the \emph{entry-function} of some other \cpp{std::thread}, on any thread other
than its own.

\ectx provides a method to test for this.
If \cpp{std::execution\_context<>::any\_thread()} returns \cpp{false}, it is
only valid to resume that \ectx instance on the thread on which it was
initially launched.


\uabschnitt{Termination}
When you explicitly construct a particular \cpp{std::execution\_context<args...>}
specialization, passing its constructor a function or lambda, that callable
must accept that same\\
\cpp{std::execution\_context<args...>} specialization as
its first parameter. It must return that
same \cpp{std::execution\_context<args...>} specialization as well.

When that toplevel callable returns a \ectx instance, the running context is
voluntarily terminated. Control switches to the context indicated by the
returned \ectx instance.

If the toplevel callable returns the same \ectx instance it was originally
passed (or rather, the most recently updated instance returned from the
original instance's \op), control returns to the context that originally
resumed the running callable. However, the callable may return (switch to)
any reachable valid \ectx instance with the correct type signature.


\uabschnitt{Exceptions}
If an uncaught exception escapes from the toplevel context function,
\cpp{std::terminate} is called.


\uabschnitt{Executing function on top of a context}
Sometimes it is useful to execute a new function (for instance, to trigger
unwinding the stack) on top of a suspended context. For this purpose you may
pass to \ectxop:

\begin{itemize}
  \item the special argument \cpp{invoke\_ontop\_arg}
  \item the function to execute
  \item any additional arguments required by the \ectx specialization.
\end{itemize}

The function passed in this case must accept as parameters the \ectx
specialization for that context plus any arguments specified by that
specialization. It must return a tuple of that \ectx specialization plus the
same set of arguments.\footnote{But in the case
of \cpp{std::execution\_context<void>}, the return type is
simply \cpp{std::execution\_context<void>}.}

For purposes of discussion, consider two \cpp{std::execution\_context<void>}
instances: \cpp{mainctx} and \cpp{fctx}. Suppose that code running
on the program's main context instantiates \cpp{fctx} with function\\
\cpp{f(std::execution\_context<void> mainctx)} and calls \cpp{fctx()}, thereby
entering \cpp{f()}. This is the point at which \cpp{mainctx} is synthesized
and passed into \cpp{f()}.

Suppose further that after doing some work, \cpp{f()} calls \cpp{mainctx()},
thereby switching context back to the main context. \cpp{f()} remains
suspended in the call to \cpp{mainctx.operator()()}.

At this point the main context calls \cpp{fctx(std::invoke\_ontop\_arg, g);}
where \cpp{g()} is declared as:
\newline
\cpp{std::execution\_context<void> g(std::execution\_context<void> gmctx);}
\newline
\cpp{g()} is entered in the context of \cpp{f()}. It is as if \cpp{f()}'s call
to \cpp{mainctx.operator()()} directly called \cpp{g()}.

However, as usual, the \cpp{fctx.operator()()} call synthesizes a \ectx
instance representing the main context and passes it to \cpp{g()}
as \cpp{gmctx}.

Function \cpp{g()} has the same range of possibilities as any function called
on \cpp{f()}'s context. It can context-switch back to the main context, or to
any other reachable context. Its special invocation only matters when control
leaves it in either of two ways:

\begin{enumerate}
  \item If \cpp{g()} throws an exception, that exception unwinds all previous
  stack entries in that context (such as \cpp{f()}'s) as well, back to a
  matching \cpp{catch} clause.
  \item If \cpp{g()} returns, its return value becomes the value returned
  by \cpp{f()}'s suspended \cpp{mainctx.operator()()} call. This is
  why \cpp{g()}'s return type must be the same as that of \op, rather than
  that of an ordinary toplevel context function.
\end{enumerate}

Consider the following example:

\cppf{ontop}

Control passes from (a) to (b) to (c), and so on.

The \cpp{ctx(std::invoke\_ontop\_arg, f2, data+1)} call at (l) passes control
to \cpp{f2()} on the context originally created for \cpp{f1()}.

The \cpp{return} statement at (n) causes the \op call at (i) to return,
executing the assignment at (o). The \cpp{std::tuple} returned by \cpp{f2()}
is directly returned to that assignment at (o).

So in this example, the call at (l) synthesizes a \ectx instance representing
the main context and passes it to \cpp{f2()} as \cpp{ctx}. \cpp{f2()} returns
that \cpp{ctx} instance, which is received by \cpp{f1()} and assigned
to \emph{its} \cpp{ctx} variable. Finally, \cpp{f1()} returns its
own \cpp{ctx} variable, switching back to the main context.


\uabschnitt{Stack destruction}
On construction of \cpp{execution\_context} a stack is allocated. If the toplevel
context-function returns, the stack will be destroyed. If the context-function
has not yet returned and the destructor of a valid \cpp{execution\_context}
instance (\cpp{execution\_context::operator bool()} returns \cpp{true}) is
called, the stack will be unwound and destroyed.\footnote{An implementation is
free to unwind the stack without throwing an exception. However, if an
exception is thrown, it should be named \cpp{std::execution\_context\_unwind}.
Portable consumer code \emph{must} permit \cpp{std::execution\_context\_unwind}
exceptions to propagate, even if all other exceptions are caught
with \cpp{catch (...)}.}


\uabschnitt{API}
declaration of class \ectx
\cppf{ec}
\paragraph*{member functions}
\subparagraph*{(constructor)}
constructs new execution context\\

\begin{tabular}{ l l }
    \midrule

    \cpp{execution\_context() noexcept} & (1)\\

    \midrule

    \cpp{template<typename StackAlloc, typename Fn, typename ... Args>}\\
    \cpp{execution\_context(std::allocator\_arg\_t, StackAlloc salloc,}\\
    \cpp{                   Fn&& fn, Args&& ... args)} & (2)\\

    \midrule

    \cpp{template<typename Fn, typename ... Args>}\\
    \cpp{explicit execution\_context(Fn&& fn, Args&& ... args)} & (3)\\

    \midrule

    \cpp{execution\_context(execution\_context&& other)} & (4)\\

    \midrule

    \cpp{execution\_context(const execution\_context& other)=delete} & (5)\\

    \midrule
\end{tabular}

\begin{description}
    \item[1)] this constructor represents \emph{not-a-context}
    \item[2)] this constructor takes (e.g.) a lambda as argument, stack is
              constructed using \emph{salloc}
    \item[3)] takes (e.g.) lambda as argument,
              stack is constructed using either \cpp{fixedsize}
              or \cpp{segmented}. An implementation may infer which of these
              best suits the code in \cpp{fn}. If it cannot
              infer, \cpp{fixedsize} will be used.
    \item[4)] moves underlying capture record to new \ectx
    \item[5)] copy constructor deleted
\end{description}

{\bfseries Notes}
\newline
When an \ectx is constructed using either of the constructors accepting
\cpp{fn}, control is \emph{not} immediately passed to \cpp{fn}. Resuming
(entering) \cpp{fn} is performed by calling \cpp{operator()()} on the new
\ectx instance.\\

\subparagraph*{(destructor)}
destroys an execution context\\

\begin{tabular}{ l l }
    \midrule

    \cpp{\~execution\_context()} & (1)\\

    \midrule
\end{tabular}

\begin{description}
    \item[1)] destroys a \ectx. If associated with a context of execution and
              holds the last reference to the internal capture record, then the
              context of execution is destroyed too. Specifically, the stack is
              unwound.\\
\end{description}

\subparagraph*{operator=}
copies/moves the context object\\

\begin{tabular}{ l l }
    \midrule

    \cpp{execution\_context& operator=(execution\_context&& other)} & (1)\\

    \midrule

    \cpp{execution\_context& operator=(const execution\_context& other)=delete} & (2)\\

    \midrule
\end{tabular}

\begin{description}
    \item[1)] assigns the state of \emph{other} to \emph{*this} using move semantics
    \item[2)] copy assignment operator deleted
\end{description}

{\bfseries Parameters}
\begin{description}
    \item[other]   another execution context to assign to this object\\
\end{description}

{\bfseries Return value}
\begin{description}
    \item[*this]
\end{description}

\subparagraph*{operator()}
resume context of execution\\

\begin{tabular}{ l l }
    \midrule

    \cpp{std::tuple<execution\_context, Args ...> operator()(Args ... args)} & (1)\\

    \midrule

    \cpp{execution\_context<void> operator()()} & (2) \\

    \midrule

    \cpp{template<typename Fn>}\\
    \cpp{std::tuple<execution\_context, Args ...>}\\
    \cpp{operator()(invoke\_ontop\_arg\_t, Fn&& fn, Args ... args)} & (3)\\

    \midrule

    \cpp{template<typename Fn>}\\
    \cpp{execution\_context<void>}\\
    \cpp{operator()(invoke\_ontop\_arg\_t, Fn&& fn)} & (4)\\

    \midrule
\end{tabular}

\begin{description}
    \item[1)] suspends the active context, resumes the execution context
    \item[2)] specialization of (1) for \cpp{execution\_context<void>}
    \item[3)] suspends the active context, resumes the execution context but
        executes \cpp{fn(args ...)} in the resumed context (e.g. on top of the
        last stack frame)
    \item[4)] specialization of (3) for \cpp{execution\_context<void>}
\end{description}

{\bfseries Parameters}
\begin{description}
    \item[... args] passed to current context  returned by the most recent call
                    to \cpp{execution\_context::operator()}\\
\end{description}

{\bfseries Return value}
\begin{description}
    \item[tuple]    of \cpp{execution\_context} and returned arguments passed to
                    the most recent call to\\ \cpp{execution\_context::operator()},
                    if any and a \cpp{execution\_context} representing the
                    context that has been suspended\\
\end{description}

{\bfseries Exceptions}
\begin{description}
    \item[1)] calls \cpp{std::terminate} if an exception escapes toplevel \cpp{fn}\\
\end{description}

{\bfseries Notes}
\newline
The \emph{prologue} preserves the execution context of the calling context as
well as stack parts like \emph{parameter list} and \emph{return
address}.\footnote{required only by some x86 ABIs} Those data are restored by
the \emph{epilogue} if the calling context is resumed.
\newline
A suspended \cpp{execution\_context} can be destroyed. Its resources will be
cleaned up at that time.
\newline
The returned \cpp{execution\_context} indicates if the suspended context has
terminated (return from context-function) via \cpp{bool operator()}.
If the returned \cpp{execution\_context} has terminated no data are transferred
in the returned tuple.


\subparagraph*{operator bool}
test context if valid\\

\begin{tabular}{ l l }
    \midrule

    \cpp{explicit operator bool() const noexcept} & (1)\\

    \midrule
\end{tabular}

\begin{description}
    \item[1)] returns \cpp{true} if \cpp{*this} refers to an valid \ectx,
              \cpp{false}\xspace otherwise
\end{description}

\subparagraph*{operator!}
test context if not valid\\

\begin{tabular}{ l l }
    \midrule

    \cpp{bool operator\!() const noexcept} & (1)\\

    \midrule
\end{tabular}

\begin{description}
    \item[1)] returns \cpp{true} if \cpp{*this} refers not to an valid \ectx,
              \cpp{false}\xspace otherwise
\end{description}


\uabschnitt{Stack allocators}
are described in P0099R0\cite{P0099R0}.

\abschnitt{Technical Specification}

\subsubsection*{std::asymmetric\_coroutine<>::pull\_type}
Defined in header \cpp{<coroutine>}.\\
\begin{tabular}{ l }
    \midrule

    \cpp{template<class T> class asymmetric_coroutine<T>::pull_type;}\\

    \midrule

    \cpp{template<class T> class asymmetric_coroutine<T&>::pull_type;}\\

    \midrule

    \cpp{template<> class asymmetric_coroutine<void>::pull_type;}\\

    \midrule
\end{tabular}
\newline
The class \pullcoro provides a mechanism to receive data values from
another execution context.\\

\paragraph*{member types\\}
\begin{tabular}{ l l l }
    \midrule

    \cpp{iterator} & std::input\_iterator & (not defined for asymmetric\_coroutine<void>::pull\_type template specialization)\\

    \midrule
\end{tabular}

\paragraph*{member functions}
\subparagraph*{(constructor)}
constructs new coroutine\\

\begin{tabular}{ l l }
    \midrule

    \cpp{pull_type();} & (1)\\

    \midrule

    \cpp{pull_type(Function&& fn);} & (2)\\

    \midrule

    \cpp{pull_type(pull_type&& other);} & (3)\\

    \midrule

    \cpp{pull_type(const pull_type& other)=delete;} & (4)\\

    \midrule
\end{tabular}

\begin{description}
    \item[1)] creates a \pullcoro which does not represent a context of execution
    \item[2)] creates a \pullcoro object and associates it with a execution
              context
    \item[3)] move constructor, constructs a \pullcoro object to represent a
              context of execution that was represented by \textit{other}, after this
              call \textit{other} no longer represents a coroutine
    \item[4)] copy constructor is deleted; coroutines are not copyable\\
\end{description}

{\bf Notes}
\newline
Return values from the \corofunction are accessible via \pullcoroget.\\
If the \corofunction throws an exception, this exception is re-thrown when the
caller returns from\\
\pullcoroop.\\

{\bf Parameters}
\begin{description}
    \item[other]  another coroutine object with which to construct this coroutine object
    \item[fn]     function to execute in the new coroutine\\
\end{description}

{\bf Exceptions}
\begin{description}
    \item[1), 3)] noexcept specification: \cpp{noexcept}
    \item[2)] \cpp{std::system_error} if the coroutine could not be started
                  - the exception may represent a implementation-specific error
                  condition; re-throw user defined exceptions from \corofunction\\
\end{description}

{\bf Example}
\cppf{fibonacci.cpp}

\subparagraph*{(destructor)}
destroys a coroutine\\

\begin{tabular}{ l l }
    \midrule

    \cpp{\~pull_type();} & (1)\\

    \midrule
\end{tabular}

\begin{description}
    \item[1)] destroys a \pullcoro. If that \pullcoro is associated with a context of execution,
              then the context of execution is destroyed too. Specifically,
              its stack is unwound.\\
\end{description}

\subparagraph*{operator=}
moves the coroutine object\\

\begin{tabular}{ l l }
    \midrule

    \cpp{pull_type & operator=(pull_type&& other);} & (1)\\

    \midrule

    \cpp{pull_type & operator=(const pull_type& other)=delete;} & (2)\\

    \midrule
\end{tabular}

\begin{description}
    \item[1)] assigns the state of \textit{other} to *this using move semantics
    \item[2)] copy assignment is deleted; coroutines are not copyable\\
\end{description}

{\bf Parameters}
\begin{description}
    \item[other]   another coroutine object to assign to this coroutine object\\
\end{description}

{\bf Return value}
\begin{description}
    \item[*this]
\end{description}

{\bf Exceptions}
\begin{description}
    \item[1)] noexcept specification: \cpp{noexcept}\\
\end{description}

\subparagraph*{operator bool}
indicates whether context of execution is still valid and a return value can be
retrieved, or \corofunction has finished\\

\begin{tabular}{ l l }
    \midrule

    \cpp{operator bool();} & (1)\\

    \midrule
\end{tabular}

\begin{description}
    \item[1)] evaluates to true if coroutine is not complete (\corofunction has
        not terminated)\\
\end{description}

{\bf Exceptions}
\begin{description}
    \item[1)] noexcept specification: \cpp{noexcept}\\
\end{description}

\subparagraph*{operator()}
jump context of execution\\

\begin{tabular}{ l l }
    \midrule

    \cpp{pull_type & operator()();} & (1)\\

    \midrule
\end{tabular}

\begin{description}
    \item[1)] transfer control of execution to \corofunction\\
\end{description}

{\bf Notes}
\newline
It is important that the coroutine is still valid (\cpp{operator bool()}
returns \cpp{true}) before calling this function, otherwise it results in
undefined behaviour.\\

{\bf Return value}
\begin{description}
    \item[*this]
\end{description}

{\bf Exceptions}
\begin{description}
    \item[1)] \cpp{std::system_error} if control of execution could not be
              transferred to other execution context - the exception may
              represent a implementation-specific error condition; re-throw
              user-defined exceptions from \corofunction\\
\end{description}

\subparagraph*{get}
accesses the current value from \corofunction\\

\begin{tabular}{ l l l }
    \midrule

    \cpp{R get();} & (1) & (member of generic template)\\

    \midrule

    \cpp{R& get();} & (2) & (member of generic template)\\

    \midrule

    \cpp{void get()=delete;} & (3) & (only for asymmetric\_coroutine<void>::pull\_type template specialization)\\

    \midrule
\end{tabular}

\begin{description}
    \item[1)] access values returned from \corofunction (if move-only, the
              value is moved, otherwise copied)
    \item[2)] access reference returned from \corofunction\\
\end{description}

{\bf Notes}
\newline
It is important that the coroutine is still valid (\cpp{operator bool()}
returns \cpp{true}) before calling this function, otherwise it results in
undefined behaviour.\\
If type T is move-only, it will be returned using move semantics. With
such a type, if you call \get a second time before calling
\cpp{operator()()}, \get will throw an exception -- see below.\\

{\bf Return value}
\begin{description}
    \item[R] return type is defined by coroutine's template argument
    \item[void] coroutine does not support \get\\
\end{description}

{\bf Exceptions}
\begin{description}
    \item[1)] Once a particular move-only value has already been
        retrieved by \get, any subsequent \get call throws
        \cpp{std::coroutine_error} with an error-code
        \cpp{std::coroutine_errc::no_data} until \cpp{operator()()} is called.\\
\end{description}

\subparagraph*{swap}
swaps two coroutine objects\\

\begin{tabular}{ l l }
    \midrule

    \cpp{void swap(pull_type& other);} & (1)\\

    \midrule
\end{tabular}

\begin{description}
    \item[1)] exchanges the underlying context of execution of two coroutine
              objects\\
\end{description}

{\bf Exceptions}
\begin{description}
    \item[1)] noexcept specification: \cpp{noexcept}\\
\end{description}

\paragraph*{non-member functions}
\subparagraph*{std::swap}
Specializes \cpp{std::swap} for \pullcoro and swaps the underlying context of
lhs and rhs.\\

\begin{tabular}{ l l }
    \midrule

    \cpp{void swap(pull_type& lhs,pull_type& rhs);} & (1)\\

    \midrule
\end{tabular}

\begin{description}
    \item[1)] exchanges the underlying context of execution of two coroutine
              objects by calling \cpp{lhs.swap(rhs)}.\\
\end{description}

{\bf Exceptions}
\begin{description}
    \item[1)] noexcept specification: \cpp{noexcept}\\
\end{description}

\subparagraph*{std::begin}
Specializes \cpp{std::begin} for \pullcoro.\\

\begin{tabular}{ l l }
    \midrule

    \cpp{template<class R> asymmetric_coroutine<R>::pull_type::iterator begin(coroutine<R>::pull_type& c);} & (1)\\

    \midrule
\end{tabular}

\begin{description}
    \item[1)] creates and returns a \cpp{std::input_iterator}\\
\end{description}

\subparagraph*{std::end}
Specializes \cpp{std::end} for \pullcoro.\\

\begin{tabular}{ l l }
    \midrule

    \cpp{template<class R> asymmetric_coroutine<R>::pull_type::iterator end(coroutine<R>::pull_type& c);} & (1)\\

    \midrule
\end{tabular}

\begin{description}
    \item[1)] creates and returns a \cpp{std::input_iterator} indicating the termination of the \corofunction\\
\end{description}

Incrementing the iterator switches the execution context.
\newline
When a main-context calls \cpp{iterator::operator++()} on an iterator obtained
from an explicitly-instantiated\\
\pullcoro, it must compare the incremented
value with the iterator returned by \cpp{std::end()}. If they are unequal, the
\corofunction has passed a new data value, which can be accessed via
\cpp{iterator::operator*()}. Otherwise the \corofunction has terminated and the
incremented iterator has become invalid.\\
When a \pushcoro's \corofunction calls \cpp{iterator::operator++()} on an iterator
obtained from the \pullcoro passed by the library, control is transferred back
to the main-context. The main-context might never pass another data value. From
the \corofunction's point of view, the \cpp{iterator::operator++()} call might
never return. If it does return, the main-context has passed a new data value,
which can be accessed via \cpp{iterator::operator*()}.\\
A function written to compare the incremented iterator with the iterator
returned by \cpp{std::end()} can be used in either situation.\\
If the return-type is move-only the first call to \cpp{iterator::operator*()}
moves the value. After that, any subsequent call to \cpp{iterator::operator*()} throws an
exception (\cpp{std::coroutine_error}) until \cpp{iterator::operator++()} is called.\\
The iterator is forward-only.\\

{\bf Notes}
\newline
Because \pullcoroiterator is an \cpp{InputIterator}, you cannot expect to copy
an iterator and increment it independently of the original.\\

{\bf Example}
\cppf{input_iterator.cpp}


\subsubsection*{std::asymmetric\_coroutine<>::push\_type}
Defined in header \cpp{<coroutine>}.\\
\begin{tabular}{ l }
    \midrule

    \cpp{template<class T> class asymmetric_coroutine<T>::push_type;}\\

    \midrule

    \cpp{template<class T> class asymmetric_coroutine<T&>::push_type;}\\

    \midrule

    \cpp{template<> class asymmetric_coroutine<void>::push_type;}\\

    \midrule
\end{tabular}
\newline
The class \pushcoro provides a mechanism to send a data value from one
execution context to another.\\

\paragraph*{member types\\}
\begin{tabular}{ l l l }
    \midrule

    \cpp{iterator} & std::output\_iterator & (not defined for asymmetric\_coroutine<void>::push\_type template specialization)\\

    \midrule
\end{tabular}

\paragraph*{member functions}
\subparagraph*{(constructor)}
constructs new coroutine\\

\begin{tabular}{ l l }
    \midrule

    \cpp{push_type();} & (1)\\

    \midrule

    \cpp{push_type(Function&& fn);} & (2)\\

    \midrule

    \cpp{push_type(push_type&& other);} & (3)\\

    \midrule

    \cpp{push_type(const push_type& other)=delete;} & (4)\\

    \midrule
\end{tabular}

\begin{description}
    \item[1)] creates a \pushcoro which does not represent a context of
              execution
    \item[2)] creates a \pushcoro object and associates it with a execution
              context
    \item[3)] move constructor, constructs a \pushcoro object to represent a
              context of execution that was represented by \textit{other}, after this
              call \textit{other} no longer represents a coroutine
    \item[4)] copy constructor is deleted; coroutines are not copyable\\
\end{description}

{\bf Parameters}
\begin{description}
    \item[other] another coroutine object with which to construct this coroutine object
    \item[fn]    function to execute in the new coroutine\\
\end{description}

{\bf Exceptions}
\begin{description}
    \item[1), 3)] noexcept specification: \cpp{noexcept}
    \item[2)]    \cpp{std::system_error} if the coroutine could not be started
                  - the exception may represent a implementation-specific error
                  condition\\
\end{description}

{\bf Notes}
\newline
If the \corofunction throws an exception, this exception is re-thrown when the caller
returns from\\
\pushcoroop.\\

{\bf Example}
\cppf{access_params_asym.cpp}

\subparagraph*{(destructor)}
destroys a coroutine\\

\begin{tabular}{ l l }
    \midrule

    \cpp{\~push_type();} & (1)\\

    \midrule
\end{tabular}

\begin{description}
    \item[1)] destroys a \pushcoro. If that \pushcoro is associated with a context of execution,
              then the context of execution is destroyed too. Specifically,
              its stack is unwound.\\
\end{description}

\subparagraph*{operator=}
moves the coroutine object\\

\begin{tabular}{ l l }
    \midrule

    \cpp{push_type & operator=(push_type&& other);} & (1)\\

    \midrule

    \cpp{push_type & operator=(const push_type& other)=delete;} & (2)\\

    \midrule
\end{tabular}

\begin{description}
    \item[1)] assigns the state of \textit{other} to *this using move semantics
    \item[2)] copy assignment operator is deleted; coroutines are not copyable\\
\end{description}

{\bf Parameters}
\begin{description}
    \item[other]   another coroutine object to assign to this coroutine object\\
\end{description}

{\bf Return value}
\begin{description}
    \item[*this]
\end{description}

{\bf Exceptions}
\begin{description}
    \item[1)] noexcept specification: \cpp{noexcept}\\
\end{description}

\subparagraph*{operator bool}
indicates if context of execution is still valid, that is, \corofunction has not
finished\\

\begin{tabular}{ l l }
    \midrule

    \cpp{operator bool();} & (1)\\

    \midrule
\end{tabular}

\begin{description}
    \item[1)] evaluates to true if coroutine is not complete (\corofunction has
        not terminated)\\
\end{description}

{\bf Exceptions}
\begin{description}
    \item[1)] noexcept specification: \cpp{noexcept}\\
\end{description}

\subparagraph*{operator()}
jump context of execution\\

\begin{tabular}{ l l l }
    \midrule

    \cpp{push_type & operator()(const Arg& arg);} & (1) & (member of generic template)\\

    \midrule

    \cpp{push_type & operator()(Arg&& arg);} & (2) & (member of generic template)\\

    \midrule

    \cpp{push_type & operator()(Arg& arg);} & (3) & (member only of asymmetric\_coroutine<Arg\&>::push\_type\\
                                            &     & template specialization)\\

    \midrule

    \cpp{push_type & operator()();} & (4) & (member only of asymmetric\_coroutine<void>::push\_type\\
                                    &     & template specialization)\\

    \midrule
\end{tabular}

\begin{description}
    \item[1),2)] If \textit{Arg} is move-only, it will be passed using
        move semantics. Otherwise it will be copied.\\
\end{description}

Switches the context of execution, transferring \textit{arg} to \corofunction.\\

{\bf Note}
\newline
It is important that the coroutine is still valid (\cpp{operator bool()}
returns \cpp{true}) before calling this function, otherwise it results in
undefined behaviour.\\

{\bf Parameters}
\begin{description}
    \item[arg] argument to pass to the \corofunction\\
\end{description}

{\bf Return value}
\begin{description}
    \item[*this]
\end{description}

{\bf Exceptions}
\begin{description}
    \item[1)] \cpp{std::system_error} if control of execution could not be
              transferred to other execution context - the exception may
              represent a implementation-specific error condition; re-throw
              user-defined exceptions from \corofunction\\
\end{description}

\subparagraph*{swap}
swaps two coroutine objects\\

\begin{tabular}{ l l }
    \midrule

    \cpp{void swap(push_type& other);} & (1)\\

    \midrule
\end{tabular}

\begin{description}
    \item[1)] exchanges the underlying context of execution of two coroutine objects\\
\end{description}

{\bf Exceptions}
\begin{description}
    \item[1)] noexcept specification: \cpp{noexcept}\\
\end{description}

\paragraph*{non-member functions}
\subparagraph*{std::swap}
Specializes \cpp{std::swap} for \pushcoro and swaps the underlying context of
lhs and rhs.\\

\begin{tabular}{ l l }
    \midrule

    \cpp{void swap(push_type& lhs,push_type& rhs);} & (1)\\

    \midrule
\end{tabular}

\begin{description}
    \item[1)] exchanges the underlying context of execution of two coroutine
              objects by calling \cpp{lhs.swap(rhs)}.\\
\end{description}

{\bf Exceptions}
\begin{description}
    \item[1)] noexcept specification: \cpp{noexcept}\\
\end{description}

\subparagraph*{std::begin}
Specializes \cpp{std::begin} for \pushcoro.\\

\begin{tabular}{ l l }
    \midrule

    \cpp{template<class R> asymmetric_coroutine<R>::push_type::iterator begin(coroutine<R>::push_type& c);} & (1)\\

    \midrule
\end{tabular}

\begin{description}
    \item[1)] creates and returns a \cpp{std::output_iterator}\\
\end{description}

\subparagraph*{std::end}
Specializes \cpp{std::end} for \pushcoro.\\

\begin{tabular}{ l l }
    \midrule

    \cpp{template<class R> asymmetric_coroutine<R>::push_type::iterator end(coroutine<R>::push_type& c);} & (1)\\

    \midrule
\end{tabular}

\begin{description}
    \item[1)] creates and returns a \cpp{std::output_iterator} indicating the termination of the coroutine\\
\end{description}

Calling \cpp{iterator::operator*(Arg&&)} switches the execution context and transfers the given data value.\\
\cpp{iterator::operator*(Arg&&)} returns if other context has transferred control of execution back.\\
The iterator is forward-only.\\

{\bf Notes}
\newline
Because \pushcoroiterator is an \cpp{OutputIterator}, you cannot expect to
copy an iterator and increment it independently of the original.\\

{\bf Example}
\cppf{output_iterator.cpp}


\subsubsection*{std::symmetric\_coroutine<>::call\_type}
Defined in header \cpp{<coroutine>}.\\
\begin{tabular}{ l }
    \midrule

    \cpp{template<class T> class symmetric_coroutine<T>::call_type;}\\

    \midrule

    \cpp{template<class T> class symmetric_coroutine<T&>::call_type;}\\

    \midrule

    \cpp{template<> class symmetric_coroutine<void>::call_type;}\\

    \midrule
\end{tabular}
\newline
The class \callcoro provides a mechanism to send a data value from one
execution context to another.\\

\paragraph*{member functions}
\subparagraph*{(constructor)}
constructs new coroutine\\

\begin{tabular}{ l l }
    \midrule

    \cpp{call_type();} & (1)\\

    \midrule

    \cpp{call_type(Function&& fn);} & (2)\\

    \midrule

    \cpp{call_type(call_type&& other);} & (3)\\

    \midrule

    \cpp{call_type(const call_type& other)=delete;} & (4)\\

    \midrule
\end{tabular}

\begin{description}
    \item[1)] creates a \callcoro which does not represent a context of execution
    \item[2)] creates a \callcoro object and associates it with a execution
              context
    \item[3)] move constructor, constructs a \callcoro object to represent a
              context of execution that was represented by \textit{other}, after this
              call \textit{other} no longer represents a coroutine
    \item[4)] copy constructor is deleted; coroutines are not copyable\\
\end{description}

{\bf Notes}
\newline
If the \corofunction throws an exception and this exception is uncatched, 
\cpp{std::terminate()} is called.\\

{\bf Parameters}
\begin{description}
    \item[other]  another coroutine object with which to construct this coroutine object
    \item[fn]     function to execute in the new coroutine\\
\end{description}

{\bf Exceptions}
\begin{description}
    \item[1), 3)] noexcept specification: \cpp{noexcept}
    \item[2)] \cpp{std::system_error} if the coroutine could not be started
                  - the exception may represent a implementation-specific error
                  condition; re-throw user defined exceptions from \corofunction\\
\end{description}

{\bf Example}
\cppf{access_params_sym.cpp}

\subparagraph*{(destructor)}
destroys a coroutine\\

\begin{tabular}{ l l }
    \midrule

    \cpp{\~call_type();} & (1)\\

    \midrule
\end{tabular}

\begin{description}
    \item[1)] destroys a \callcoro. If that \callcoro is associated with a context of execution,
              then the context of execution is destroyed too. Specifically,
              its stack is unwound.\\
\end{description}

\subparagraph*{operator=}
moves the coroutine object\\

\begin{tabular}{ l l }
    \midrule

    \cpp{call_type & operator=(call_type&& other);} & (1)\\

    \midrule

    \cpp{call_type & operator=(const call_type& other)=delete;} & (2)\\

    \midrule
\end{tabular}

\begin{description}
    \item[1)] assigns the state of \textit{other} to *this using move semantics
    \item[2)] copy assignment is deleted; coroutines are not copyable\\
\end{description}

{\bf Parameters}
\begin{description}
    \item[other]   another coroutine object to assign to this coroutine object\\
\end{description}

{\bf Return value}
\begin{description}
    \item[*this]
\end{description}

{\bf Exceptions}
\begin{description}
    \item[1)] noexcept specification: \cpp{noexcept}\\
\end{description}

\subparagraph*{operator bool}
indicates whether context of execution is still valid or \corofunction has finished\\

\begin{tabular}{ l l }
    \midrule

    \cpp{operator bool();} & (1)\\

    \midrule
\end{tabular}

\begin{description}
    \item[1)] evaluates to true if coroutine is not complete (\corofunction has
        not terminated)\\
\end{description}

{\bf Exceptions}
\begin{description}
    \item[1)] noexcept specification: \cpp{noexcept}\\
\end{description}

\subparagraph*{operator()}
jump context of execution\\

\begin{tabular}{ l l l }
    \midrule

    \cpp{call_type & operator()(const Arg& arg);} & (1) & (member of generic template)\\

    \midrule

    \cpp{call_type & operator()(Arg&& arg);} & (2) & (member of generic template)\\

    \midrule

    \cpp{call_type & operator()(Arg& arg);} & (3) & (member only of symmetric\_coroutine<Arg\&>::call\_type\\
                                            &     & template specialization)\\

    \midrule

    \cpp{call_type & operator()();} & (4) & (member only of symmetric\_coroutine<void>::call\_type\\
                                    &     & template specialization)\\

    \midrule
\end{tabular}

\begin{description}
    \item[1),2)] If \textit{Arg} is move-only, it will be passed using
        move semantics. Otherwise it will be copied.\\
\end{description}

Switches the context of execution, transferring \textit{arg} to \corofunction.\\

{\bf Notes}
\newline
It is important that the coroutine is still valid (\cpp{operator bool()}
returns \cpp{true}) before calling this function, otherwise it results in
undefined behaviour.\\

{\bf Return value}
\begin{description}
    \item[*this]
\end{description}

{\bf Exceptions}
\begin{description}
    \item[1)] noexcept specification: \cpp{noexcept}\\
\end{description}

\subparagraph*{swap}
swaps two coroutine objects\\

\begin{tabular}{ l l }
    \midrule

    \cpp{void swap(call_type& other);} & (1)\\

    \midrule
\end{tabular}

\begin{description}
    \item[1)] exchanges the underlying context of execution of two coroutine
              objects\\
\end{description}

{\bf Exceptions}
\begin{description}
    \item[1)] noexcept specification: \cpp{noexcept}\\
\end{description}

\paragraph*{non-member functions}
\subparagraph*{std::swap}
Specializes \cpp{std::swap} for \callcoro and swaps the underlying context of
lhs and rhs.\\

\begin{tabular}{ l l }
    \midrule

    \cpp{void swap(call_type& lhs,call_type& rhs);} & (1)\\

    \midrule
\end{tabular}

\begin{description}
    \item[1)] exchanges the underlying context of execution of two coroutine
              objects by calling \cpp{lhs.swap(rhs)}.\\
\end{description}

{\bf Exceptions}
\begin{description}
    \item[1)] noexcept specification: \cpp{noexcept}\\
\end{description}


\subsubsection*{std::symmetric\_coroutine<>::yield\_type}
Defined in header \cpp{<coroutine>}.\\
\begin{tabular}{ l }
    \midrule

    \cpp{template<class T> class symmetric_coroutine<T>::yield_type;}\\

    \midrule

    \cpp{template<class T> class symmetric_coroutine<T&>::yield_type;}\\

    \midrule

    \cpp{template<> class symmetric_coroutine<void>::yield_type;}\\

    \midrule
\end{tabular}
\newline
The class \yieldcoro provides a mechanism to receive data values from
another execution context and to transfer the execution control to another
coroutine.\\

\paragraph*{member functions}
\subparagraph*{(constructor)}
constructs new coroutine\\

\begin{tabular}{ l l }
    \midrule

    \cpp{yield_type();} & (1)\\

    \midrule

    \cpp{yield_type(yield_type&& other);} & (2)\\

    \midrule

    \cpp{yield_type(const yield_type& other)=delete;} & (3)\\

    \midrule
\end{tabular}

\begin{description}
    \item[1)] creates a \yieldcoro which does not represent a context of execution
    \item[2)] move constructor, constructs a \yieldcoro object to represent a
              context of execution that was represented by \textit{other}, after this
              call \textit{other} no longer represents a coroutine
    \item[3)] copy constructor is deleted; coroutines are not copyable\\
\end{description}

{\bf Notes}
\newline
Values to the \corofunction are accessible via \yieldcoroget.\\
\yieldcoro can be synthesized by the library only.\\

{\bf Parameters}
\begin{description}
    \item[other]  another coroutine object with which to construct this coroutine object
\end{description}

{\bf Exceptions}
\begin{description}
    \item[1) - 3)] noexcept specification: \cpp{noexcept}
\end{description}

\subparagraph*{(destructor)}
destroys a coroutine\\

\begin{tabular}{ l l }
    \midrule

    \cpp{\~yield_type();} & (1)\\

    \midrule
\end{tabular}

\begin{description}
    \item[1)] destroys a \yieldcoro
\end{description}

\subparagraph*{operator=}
moves the coroutine object\\

\begin{tabular}{ l l }
    \midrule

    \cpp{yield_type & operator=(yield_type&& other);} & (1)\\

    \midrule

    \cpp{yield_type & operator=(const yield_type& other)=delete;} & (2)\\

    \midrule
\end{tabular}

\begin{description}
    \item[1)] assigns the state of \textit{other} to *this using move semantics
    \item[2)] copy assignment is deleted; coroutines are not copyable\\
\end{description}

{\bf Parameters}
\begin{description}
    \item[other]   another coroutine object to assign to this coroutine object\\
\end{description}

{\bf Return value}
\begin{description}
    \item[*this]
\end{description}

{\bf Exceptions}
\begin{description}
    \item[1)] noexcept specification: \cpp{noexcept}\\
\end{description}

\subparagraph*{operator bool}
indicates whether the coroutine is still a valid instance\\

\begin{tabular}{ l l }
    \midrule

    \cpp{operator bool();} & (1)\\

    \midrule
\end{tabular}

\begin{description}
    \item[1)] evaluates to true if the instance is a valid coroutine\\
\end{description}

{\bf Exceptions}
\begin{description}
    \item[1)] noexcept specification: \cpp{noexcept}\\
\end{description}

\subparagraph*{operator()}
jump context of execution\\

\begin{tabular}{ l l }
    \midrule

    \cpp{yield_type & operator()();} & (1)\\

    \midrule

    \cpp{template< typename X > yield_type & operator()( symmetric_coroutine< T >::call_type & other, X & x);} & (2)\\

    \midrule

    \cpp{template<> yield_type & operator()( symmetric_coroutine< void >::call_type & other);} & (3)\\

    \midrule
\end{tabular}

\begin{description}
    \item[1)] transfer control of execution to the starting point, e.g invocation of\\\callcoroop
    \item[2)] transfer control of execution to another symmetric coroutine, parameter \cpp{x} is passed as value into other's context
    \item[3)] transfer control of execution to another symmetric coroutine\\
\end{description}

{\bf Notes}
\newline
It is important that the coroutine is still valid (\cpp{operator bool()}
returns \cpp{true}) before calling this function, otherwise it results in
undefined behaviour.\\

{\bf Return value}
\begin{description}
    \item[*this]
\end{description}

{\bf Exceptions}
\begin{description}
    \item[1)] \cpp{std::system_error} if control of execution could not be
              transferred to other execution context - the exception may
              represent a implementation-specific error condition; re-throw
              user-defined exceptions from \corofunction\\
\end{description}

{\bf Example}
\cppf{jump_sym.cpp}

\subparagraph*{get}
accesses the current value passed to \corofunction\\

\begin{tabular}{ l l l }
    \midrule

    \cpp{R get();} & (1) & (member of generic template)\\

    \midrule

    \cpp{R& get();} & (2) & (member of generic template)\\

    \midrule

    \cpp{void get()=delete;} & (3) & (only for symmetric\_coroutine<void>::yield\_type template specialization)\\

    \midrule
\end{tabular}

\begin{description}
    \item[1)] access values passed to \corofunction (if move-only, the
              value is moved, otherwise copied)
    \item[2)] access reference passed to \corofunction\\
\end{description}

{\bf Notes}
\newline
If type T is move-only, it will be returned using move semantics. With
such a type, if you call \get a second time before calling
\cpp{operator()()}, \get will throw an exception -- see below.\\

{\bf Return value}
\begin{description}
    \item[R] return type is defined by coroutine's template argument
    \item[void] coroutine does not support \get\\
\end{description}

{\bf Exceptions}
\begin{description}
    \item[1)] Once a particular move-only value has already been
        retrieved by \get, any subsequent \get call throws
        \cpp{std::coroutine_error} with an error-code
        \cpp{std::coroutine_errc::no_data} until \cpp{operator()()} is called.\\
\end{description}

\subparagraph*{swap}
swaps two coroutine objects\\

\begin{tabular}{ l l }
    \midrule

    \cpp{void swap(yield_type& other);} & (1)\\

    \midrule
\end{tabular}

\begin{description}
    \item[1)] exchanges the underlying context of execution of two coroutine
              objects\\
\end{description}

{\bf Exceptions}
\begin{description}
    \item[1)] noexcept specification: \cpp{noexcept}\\
\end{description}

\paragraph*{non-member functions}
\subparagraph*{std::swap}
Specializes \cpp{std::swap} for \yieldcoro and swaps the underlying context of
lhs and rhs.\\

\begin{tabular}{ l l }
    \midrule

    \cpp{void swap(yield_type& lhs,yield_type& rhs);} & (1)\\

    \midrule
\end{tabular}

\begin{description}
    \item[1)] exchanges the underlying context of execution of two coroutine
              objects by calling \cpp{lhs.swap(rhs)}.\\
\end{description}

{\bf Exceptions}
\begin{description}
    \item[1)] noexcept specification: \cpp{noexcept}\\
\end{description}


\subsubsection*{std::coroutine\_errc}
Defined in header \cpp{<coroutine>}.\\

\begin{tabular}{ l }
    \midrule

    \cpp{enum class coroutine_errc \{ no_data \};}\\

    \midrule
\end{tabular}

Enumeration \cpp{std::coroutine_errc} defines the error codes reported by
\pullcoro or \yieldcoro in \cpp{std::coroutine_error} exception object.

\paragraph*{member constants}
Determines error code.\\

\begin{tabular}{ l l }
    \midrule

    \cpp{no_data} & \pullcoro or \yieldcoro has no\\ & valid data (maybe moved by prior access)\\

    \midrule
\end{tabular}
\newline


\subsubsection*{std::coroutine\_error}
Defined in header \cpp{<coroutine>}.\\

\begin{tabular}{ l }
    \midrule

    \cpp{class coroutine_error;}\\

    \midrule
\end{tabular}

The class \cpp{std::coroutine_error} defines an exception class that is derived
from \cpp{std::logic_error}.

\paragraph*{member functions}
\subparagraph*{(constructor)}
constructs new coroutine error object.\\

\begin{tabular}{ l l }
    \midrule

    \cpp{coroutine_error( std::error_code ec);} & (1)\\

    \midrule
\end{tabular}

\begin{description}
    \item[1)] creates a \cpp{std::coroutine_error} error object from an error-code.\\
\end{description}

{\bf Parameters}
\begin{description}
    \item[ec] error-code
\end{description}

\subparagraph*{code}
Returns the error-code.\\

\begin{tabular}{ l l }
    \midrule

    \cpp{const std::error_code& code() const;} & (1)\\

    \midrule
\end{tabular}

\begin{description}
    \item[1)] returns the stored error code.\\
\end{description}

{\bf Return value}
\begin{description}
    \item[std::error\_code] stored error code\\
\end{description}

{\bf Exceptions}
\begin{description}
    \item[1)] noexcept specification: \cpp{noexcept}\\
\end{description}

\subparagraph*{what}
Returns a error-description.\\

\begin{tabular}{ l l }
    \midrule

    \cpp{virtual const char* what() const;} & (1)\\

    \midrule
\end{tabular}

\begin{description}
    \item[1)] returns a description of the error.\\
\end{description}

{\bf Return value}
\begin{description}
    \item[char*] null-terminated string with error description\\
\end{description}

{\bf Exceptions}
\begin{description}
    \item[1)] noexcept specification: \cpp{noexcept}\\
\end{description}


%//////////////////////////////////////////////////////////////////////////////

\addcontentsline{toc}{subsection}{References}
\begin{thebibliography}{99}

    \bibitem{N3985}
        \href{http://www.open-std.org/jtc1/sc22/wg21/docs/papers/2014/n3985.pdf}
        {N3985: A proposal to add coroutines to the C++ standard library}

    \bibitem{N4232}
        \href{http://www.open-std.org/jtc1/sc22/wg21/docs/papers/2014/n4232.pdf}
        {N4232: Stackful Coroutines and Stackless Resumable Functions}

    \bibitem{N4397}
        \href{http://www.open-std.org/jtc1/sc22/wg21/docs/papers/2015/n4397.pdf}
        {N4397: A low-level API for stackful coroutines}

    \bibitem{N4453}
        \href{http://www.open-std.org/jtc1/sc22/wg21/docs/papers/2015/n4453.pdf}
        {N4453: Resumable Expressions}

    \bibitem{P0057}
        \href{http://www.open-std.org/jtc1/sc22/wg21/docs/papers/2016/p0057r5.pdf}
        {P0057R5: Wording for Coroutines}

    \bibitem{P0099R0}
        \href{http://www.open-std.org/jtc1/sc22/wg21/docs/papers/2015/p0099r0.pdf}
        {P0099R0: A low-level API for stackful context switching}

    \bibitem{P0112}
        \href{http://www.open-std.org/jtc1/sc22/wg21/docs/papers/2015/p0112r0.html}
        {P0112R0: Networking Library Proposal (Revision 6)}

    \bibitem{P0158}
        \href{http://www.open-std.org/jtc1/sc22/wg21/docs/papers/2015/p0158r0.html}
        {P0158R0: Coroutines belong in a TS}

    \bibitem{gccsplit}
        \href{http://gcc.gnu.org/wiki/SplitStacks}
        {Split Stacks / GCC}

    \bibitem{basio}
        \href{http://www.boost.org/doc/libs/release/doc/html/boost\_asio.html}
        {Library \emph{Boost.Asio}}

    \bibitem{basiocoro}
        \href{http://www.boost.org/doc/libs/release/doc/html/boost_asio/reference/coroutine.html}
        {\emph{Boost.Asio} Coroutines}

    \bibitem{bcontext}
        \href{http://www.boost.org/doc/libs/release/libs/context/doc/html/index.html}
        {Library \emph{Boost.Context}}

    \bibitem{bcoroutine2}
        \href{http://www.boost.org/doc/libs/release/libs/coroutine2/doc/html/index.html}
        {Library \emph{Boost.Coroutine2}}

    \bibitem{bfiber}
        \href{http://www.boost.org/doc/libs/release/libs/fiber/doc/html/index.html}
        {Library \emph{Boost.Fiber}}

    \bibitem{boostcon}
        \href{https://www.youtube.com/watch?v=gcNphOWuUb0}
        {C++Now 2016: Nat Goodspeed, \emph{The Fiber Library}}

    \bibitem{cppcon}
        \href{https://www.youtube.com/watch?v=e-NUmyBou8Q}
        {CppCon 2016: Nat Goodspeed, \emph{Elegant Asynchronous Code}}

\end{thebibliography}


%//////////////////////////////////////////////////////////////////////////////

\abschnitt{A. \textit{jump}-operation for SYSV ABI on x86\_64}\label{appendix}

The assembler code (from \boostcontext) shows what the \textit{jump}-operation
might look like for SYSV ABI on x86\_64.
\asmf{jump.S}
Register \asm{rdi} contains a reference to the \cblock of the current execution
context \textit{X} and register \asm{rsi} points to the \cblock of the execution
context \textit{Y} which has to be resumed.\\
\newline
In lines 2-7 the content of the current non-volatile registers are stored in the
\cblock of \textit{X}.\\
Line 9 calculates the stack-pointer writes it to the \cblock in line 10.\\
Lines 11 and 12 do the same for the return address (will be assigned to the
instruction-pointer \asm{rip}).\\
\newline
The next block (lines 14-19) restore the content of non-volatile register for
the execution context \textit{Y}.\\
The stack-pointer is restored in line 21.\\
Line 22 moves the address of the instruction which should be executed in
\textit{Y} to register \asm{rcx}.\\
\newline
Lines 24 and 25 allow to transfer data (as return value in \textit{Y}) between
context jumps.\\
\newline
The next line transfers execution control (\textit{branch-and-link}) to
\textit{Y} by executing the instruction stored in \asm{rcx}.


%//////////////////////////////////////////////////////////////////////////////

\end{document}
