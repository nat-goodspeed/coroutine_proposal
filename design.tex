\abschnitt{Design Decisions}

\subsubsection*{Proposed Design}

\paragraph*{move-only:}
Each instance of \coro has its own stack and is moveable-only.\\
\newline
If \coro would be copyable then its stack with all the objects allocated on it
are copied too. This could force undefined behaviour if some of this objects are
RAII-classes (manage a resource via RAII pattern). If one of the \coro copies
terminates (unwinds its stack) the RAII-classes releases their managed resources
at destruction.

\paragraph*{clean-up:}
On destruction \coro unwinds its stack.\\
\newline
The implementer is free to deallocate the stack or cache it for future \coro
instances.

\paragraph*{context switch:}
\coro stores and restores registers according to the underlying ABI on each
context switch.\\
\newline
This includes also the floating point environment as required by the ABI. The
implementer can omit preserving the floating point env if he can predict that
it's safe.\\
\newline
On POSIX systems \coro must not preserve signal masks for performance reasons.\\
\newline
A context switch is done via \coroop.

\paragraph*{segmented stack:}
\coro must support segmented stacks (growing on demand).\\
\newline
It is not always possible to estimated the required stack size - in most cases
too much memory is allocated (wast of virtual address-space).\\
\newline
At construction \coro starts with a default (minimal) stack size. This minimal
stack size is the maximum of page size and the canonical size for signal stack
(macro SIGSTKSZ on POSIX).\\
\newline
At this time of writing only GCC (4.7) is known to support segmented
stacks. With version 1.54 \boostcoroutine provides support for segmented stacks.\\
\newline
The destructor of \coro deallocates the associated stack. The implementer is
free to deallocated the stack or to cache it for later usage.

\paragraph*{start at construction:}
\corofunction is entered upon construction of \coro not upon the first call.\\
\newline
Because \corofunction is entered upon construction of \coro we can check after
returning from the constructor if a result was provided or the \corofunction
has terminated.\\
\corobool returns true if \coro is still valid (\corofunction has not terminated)
and data is available.\\
The iteration stopped if data is no longer generated:
\cppf{corofn_startup.cpp}
Otherwise detecting the completeness of \coro would require an extra context
switch (return from last \coroop and calling return in the \corofunction).\\
This makes the implementation for iterators very hard and \cpp{std::distance}
would return 4 instead of 3 in the example above.\\
\newline
If parameters are required to be passed at the start of \corofunction then those
parameters can be passed via the \coro constructor.

\paragraph*{coroutine-function signature:}
The \corofunction returns \cpp{void} and takes a \coro of inverted signature
as argument.\\
\newline
This design decision makes the code using \coro let look more symmetric.\\
Both, calling code and coroutine function, use an object of type \coro in order
to switch the context and to transfer data.\\
This requires that the template parameter of the coroutine passed as argument
to the \corofunction has an inverted signature, e.g. return types become
arguments and vice versa.
\cppf{corofn_startup.cpp}

\coro in the main code has a signature of \cpp{int(void)} - it returns an
integer from \corofunction.\\
The \coro argument of \corofunction \cpp{f} has therefore a signature of
\cpp{void(int)} - it transfers an integer to caller.\\
\newline
\corofunction must return void, otherwise the user would call\\
\coroop to return values from the \corofunction and \cpp{return} as the
last call which terminates the \corofunction - with the chosen design we have
only one path to exit the \corofunction.
\cppf{corofn_nonvoid.cpp}

\paragraph*{passing parameters:}
Parameters are passed to the \corofunction via\\\coroop.\\
\newline
\coroop accepts arguments as defined in the template arguments in \coro. A
context switch is executed and the passed parameters will be accessible in in
the \corofunction.

\paragraph*{accessing parameters:}
Parameters returned from the \corofunction can be accessed with\\\coroget.\\
\newline
In contrast to single argument in coroutine signature multiple arguments are
returned via \tuple from \coroget.\\
Splitting-up the access of parameters from context switch function enables to
check if \coro is valid after return from \coroop, e.g. \coro has values and
\corofunction has not terminated.
\cppf{access_params.cpp}

\paragraph*{returning parameters:}
Results can be returned by a \corofunction via\\\coroop.\\
\newline
Like passing parameters to the \corofunction the \corofunction can return
results to the caller.\\ This is done by calling the \coroop of the \coro given
as first argument to \corofunction.
\cppf{return_params.cpp}

\paragraph*{output iterator:}
\coro with signature \cpp{T(void)} provide output iterators.
\cppf{output_iterator.cpp}

\paragraph*{input iterator:}
\coro with signature \cpp{void(T)} provide input iterators.

\paragraph*{exceptions:}
An exception thrown inside \corofunction will re-thrown by \coro constructor or
\coroop.

\subsubsection*{Alternative Design}
\boostcoroutine is a follow-up on \boostcorosum which is unfinished and
not part of the official boost release (development of this library was
stopped).\\
During the boost review process interface of \boostcoroutine was changed and
differ in some parts to \boostcorosum.
