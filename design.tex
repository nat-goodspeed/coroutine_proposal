\newpage
\abschnitt{Design}
Class \ectx is derived from the work on Boost.Context\cite{bcontext}. It
provides a {\bfseries small, basic API} on which to build {\bfseries
higher-level APIs} such as stackful coroutines (N3985\cite{N3985}, working
implementation Boost.Coroutine2\cite{bcoroutine2}) and user-mode threads
(executing tasks in a cooperative multitasking environment, working
implementation Boost.Fiber\cite{bfiber}).

In effect, \cpp{execution_context} is a copyable handle to an underlying,
noncopyable \emph{capture record}.

\cpp{execution_context::operator()} preserves the CPU register set
\footnote{defined by ABI's calling convention} + stack pointer: the content of
those registers is stored in the capture record associated with
\cpp{execution_context::current()}. Then \cpp{operator()} loads the CPU register
set and stack pointer from the capture record associated with \cpp{*this}.

Finally, \cpp{execution_context::operator()}
stores \cpp{execution_context::current()} in the capture record
associated with \cpp{*this} -- to permit backtracking through the history of
context switches. It makes \cpp{*this} become \cpp{current()}.

\uabschnitt{Class \ectx}
Based on implementation experience with \cpp{execution_context} in
Boost.Coroutine2\cite{bcoroutine2} and Boost.Fiber\cite{bfiber}, the author
noticed that \cpp{execution_context} is almost always passed a lambda as its
constructor argument. Lambda captures are especially useful for transporting
data between different execution contexts (address of lvalue, residing on
stack/heap).\\
Why not implicitly construct \ectx with a 'resumable lambda'-like
syntax\cite{N4244}, instead of explicitly passing a lambda as a constructor
argument? Even if \ectx is constructed like a lambda, actually it isn't a
lambda (it is constructible from \ectxcurrent too).\\
\ectx is more like a handle to a \emph{capture record} of an execution
context.
\cppf{N4397/rl1}
The keyword \resumable together with a hint (attribute) about the type and size
of the side stack tells the compiler to generate a stackful execution context.\\
Because of the symmetric context switching (only one operation transfers
control), the target \cpp{execution_context} must be explicitly specified.\\
Exchanging data between different execution contexts requires the use of lambda
captures.

\uabschnitt{First-class object}
As first-class object the execution context can be stored in a variable or
container.

\uabschnitt{Capture record}
Each instance of \ectx owns a toplevel activation record, the capture record.
The capture record is a special activation record that stores additional data
such as stack pointer, instruction pointer and a link to its parent execution
context. That means that during a context switch, the execution state
of the running context is captured and stored in the capture record while the
content of the resumed execution context is loaded (into CPU registers etc.).

\paragraph*{Parent context}
An \cpp{execution_context}'s pointer to
its \emph{parent} \cpp{execution_context} allows one to traverse the chain of
ancestor contexts, that is, to locate the \cpp{execution_context} which resumed
(called \ectxop on) the currently active \cpp{execution_context}. Of course
the algorithm must prevent a cycle of parents. The main context (created
on entry of \main or thread's \emph{entry-function}) will never have a parent
because if the main context terminates the application or the thread terminates.

\paragraph*{Active context}
The static member function \ectxcurrent returns a \ectx pointing to the
current capture record. The current active capture record is stored in an
internal, thread local pointer.

\paragraph*{Toplevel capture records}
On entering \main as well as the \emph{entry-function} of a thread, an execution
context (capture record) is created and stored in the internal pointer underlying
\ectxcurrent. (In practice, this could be lazily instantiated.)

\uabschnitt{Suspend-on-call}
\ectx is designed to suspend on call of \ectxop. The \emph{prologue} of \ectxop
captures (preserves) the active execution context and loads the data from the
capture record of the resumed context (\emph{*this}) into the CPU.\\
The previous running context becomes the parent of the resumed context and gets
suspended. This forms a parent-child relationship between the suspended and
the resumed context. The parent may vary between successive calls of \ectxop.

\paragraph*{Termination}
If the body of the toplevel context function reaches its end, the parent
execution context (pointer in the capture record) is resumed. That means that
in the parent context the function \ectxop returns. For this purpose the
\emph{epilogue} of \ectxop loads the capture record (instruction pointer, stack
pointer etc.) of the parent context, so that it is resumed.

\uabschnitt{Exceptions}
An uncaught exception thrown from the execution context is caught, the parent
execution context is resumed and the exception is re-thrown inside the parent
context (that is, the exception is emitted by \ectxop).
\newpage
\paragraph*{member functions}
\subparagraph*{(constructor)}
constructs new execution context\\

\begin{tabular}{ l l }
    \midrule

    {\ttfamily\small\color{black}[captures](params) {\color{blue}mutable} {\color{blue}resumable}(hint) attrs -> ret \{body\}} & (1)\\

    \midrule

    \cpp{execution_context(const execution_context& other)=default} & (2)\\

    \midrule

    \cpp{execution_context(execution_context&& other)=default} & (3)\\

    \midrule
\end{tabular}

\begin{description}
    \item[1)] the constructor does not take a lambda as argument, instead the
              compiler evaluates the lambda-like syntax and constructs a \ectx
              directly
              \begin{description}
                  \item[captures]     list of captures
                  \item[params]       only empty parameter-list allowed
                  \item[mutable]      parameters captured by copy can be modified by \emph{body}
                  \item[resumable]    identify resumable context
                  \item[hint]         stack type hint:
                                      \begin{itemize}
                                          \item \cpp{fixedsize(x=default_stacksize)}:
                                              fixed size stack (\cpp{default_stacksize} is
                                              platform dependent)
                                          \item \cpp{segmented(x=default_segmented_stacksize)}:
                                              stack grows on demand (\cpp{default_segmented_stacksize}
                                              is platform dependent)
                                      \end{itemize}
                  \item[attrs]        attributes for \cpp{operator()}
                  \item[ret]          only \cpp{void} allowed
                                      (use capture list instead)
                  \item[body]         function body\\
              \end{description}
    \item[2)] copies \ectx, e.g. underlying capture record is shared
    \item[3)] moves underlying capture record to new \ectx
\end{description}

{\bfseries Notes}
\newline
If an instance of \ectx is copied, both instances share the same underlying
capture record.

\subparagraph*{(destructor)}
destroys an execution context\\

\begin{tabular}{ l l }
    \midrule

    \cpp{\~execution_context()} & (1)\\

    \midrule
\end{tabular}

\begin{description}
    \item[1)] destroys a \ectx. If associated with a context of execution and
              holds the last reference to the internal capture record, then the
              context of execution is destroyed too. Specifically, the stack is
              unwound.\\
\end{description}

\subparagraph*{operator=}
copies/moves the coroutine object\\

\begin{tabular}{ l l }
    \midrule

    \cpp{execution_context& operator=(execution_context&& other)} & (1)\\

    \midrule

    \cpp{execution_context& operator=(const execution_context& other)} & (2)\\

    \midrule
\end{tabular}

\begin{description}
    \item[1)] assigns the state of \emph{other} to \emph{*this} using move semantics
    \item[2)] copies the state of \emph{other} to \emph{*this}, state (capture
              record) is shared
\end{description}

{\bfseries Parameters}
\begin{description}
    \item[other]   another execution context to assign to this object\\
\end{description}

{\bfseries Return value}
\begin{description}
    \item[*this]
\end{description}

\subparagraph*{operator()}
jump context of execution\\

\begin{tabular}{ l l }
    \midrule

    {\ttfamily\small\color{black}execution\_context\& {\color{blue}operator}()()} & (1)\\

    \midrule
\end{tabular}

\begin{description}
    \item[1)] suspends the active context, resumes the execution context\\
\end{description}

{\bfseries Exceptions}
\begin{description}
    \item[1)] re-throws the exception of the resumed context in the parent context\\
\end{description}

{\bfseries Notes}
\newline
An exception thrown inside execution context is caught, the parent execution
context is resumed and the exception is re-thrown in the parent context (out of
\ectxop).\\
If the toplevel context function terminates (reaches end), the parent context is
resumed (return of\\
\ectxop in the parent execution context).\\
The behaviour is undefined if \cpp{operator()} is called while \cpp{current()}
returns \emph{*this} (attempting to resume an already running context).

\subparagraph*{explicit bool operator}
test if context has not reached its end\\

\begin{tabular}{ l l }
    \midrule

    {\ttfamily\small\color{black}{\color{blue}explicit} {\color{blue}bool} {\color{blue}operator}() {\color{blue}noexcept}} & (1)\\

    \midrule
\end{tabular}

\begin{description}
    \item[1)] returns \emph{true} if context is not terminated\\
\end{description}

{\bfseries Exceptions}
\begin{description}
    \item[1)] noexcept specification: \cpp{noexcept}\\
\end{description}

\subparagraph*{operator!}
test if context has reached its end\\

\begin{tabular}{ l l }
    \midrule

    {\ttfamily\small\color{black}{\color{blue}bool} {\color{blue}operator}!() {\color{blue}noexcept}} & (1)\\

    \midrule
\end{tabular}

\begin{description}
    \item[1)] returns \emph{true} if context is terminated\\
\end{description}

{\bfseries Exceptions}
\begin{description}
    \item[1)] noexcept specification: \cpp{noexcept}\\
\end{description}

\subparagraph*{current}
accesses the current active execution context\\

\begin{tabular}{ l l }
    \midrule

    \cpp{static execution_context current()} & (1)\\

    \midrule
\end{tabular}

\begin{description}
    \item[1)] construct an instance of \ectx pointing to the capture record of
              the current, active execution context\\
\end{description}

{\bfseries Notes}
\newline
The current active execution context is thread-specific, e.g. for each thread
(including \main) an execution context is created at start-up.
