\abschnitt{Design}
Class \ectx is derived from the work on boost.context\cite{bcontext} - it
provides a small, basic API, suitable to implement high-level APIs for stackful
coroutines (N3985\cite{N3985}, boost.coroutine2\cite{bcoroutine2}) and user-mode
threads (executing tasks in a
cooperative multitasking environment, boost.fiber\cite{bfiber}).\\

\uabschnitt{Class \ectx}
Based on the implementation experience with \cpp{execution_context} in
boost.coroutine2\cite{bcoroutine2} and boost.fiber\cite{bfiber} the author
encountered that \cpp{execution_context} is almost always used together with
lambdas (passed as argument to the constructor of \ectx). Especially the
lambda captures are suitable to transport data between different execution
context's (address of lvalue, reside on stack/heap).\\
Why not construct \ectx directly with an 'resumable lambda'-like syntax?
\cppf{rl1.cpp}
The differences to N4244 are the absence of keyword \yield because of
symmetric context switching (== only one operation transfers the control of
execution), e.g. the target switch must be explicitly specified.\\
Stackful resumable lambdas require keyword \resumable together with an hint
(attribute) about the type and size of the stack.\\
Exchanging data between execution context's requires the use of lambda captures.\\
A main execution context is created at statup (entering function \main),
accessible via \ectxcurrent (same applies for threads).

\paragraph*{member functions}
\subparagraph*{(constructor)}
constructs new execution context\\

\begin{tabular}{ l l }
    \midrule

    \cpp{[capture-list] (params) mutable resumable(hint) exceptions attribute -> ret \{body\}} & (1)\\

    \midrule

    \cpp{execution_context(execution_context const& other)=default;} & (2)\\

    \midrule

    \cpp{execution_context(execution_context&& other)=default;} & (3)\\

    \midrule
\end{tabular}

\begin{description}
    \item[1)] creates a \ectx from a 'resumable lambda'-like syntax
              \begin{description}
                  \item[mutable]      allows to modify parameters captured by copy
                  \item[resumable]    identify resumable operation
                  \item[hint]         stack type hint:
                                      \begin{itemize}
                                          \item <no hint specified>: create stackless
                                              resumable lambda
                                          \item \cpp{fixedsize(x=default_stacksize)}:
                                              create stackful resumable lambda;
                                              fixed size stack (\cpp{default_stacksize)} is
                                              platform depended)
                                          \item \cpp{segmented(x=default_stacksize)}:
                                              create stackful resumable lambda;
                                              stack grows on demand (\cpp{default_stacksize)}
                                              is platform depended)
                                      \end{itemize}
                  \item[exceptions]   only \cpp{noexcept} allowed; no exception is
                                      permitted to leave the body otherwise
                                      \cpp{std::terminate()} is called
                  \item[attribute]    attributes for \cpp{operator()}
                  \item[capture-list] list of captures
                  \item[params]       only empty parameter-list allowed
                  \item[ret]          only \cpp{void} allowed; resumable lambda returns nothing
                  \item[body]         function body\\
              \end{description}
    \item[2)] copies \ectx, e.g. underlying control block is shared
    \item[3)] moves underlying control block to new \ectx
\end{description}

{\bf Notes}
\newline
If an instance of \ectx is copied, both instances share the same underlying
control block. Resuming one instance modifies the control block of the other
\ectx too.\\
If this is behaviour is not permitted, the stack has to be copied. That requires
identification and modification of local variables pointing to address of the
stack.\\

\subparagraph*{(destructor)}
destroys a execution context\\

\begin{tabular}{ l l }
    \midrule

    \cpp{\~execution_context();} & (1)\\

    \midrule
\end{tabular}

\begin{description}
    \item[1)] destroys a \ectx. If associated with a context of execution and
              holds the last reference to the internal control block, then the
              context of execution is destroyed too. Specifically, the stack is
              unwound.\\
\end{description}

\subparagraph*{operator=}
copies/moves the coroutine object\\

\begin{tabular}{ l l }
    \midrule

    \cpp{execution_context & operator=(execution_context&& other);} & (1)\\

    \midrule

    \cpp{execution_context & operator=(const execution_context& other);} & (2)\\

    \midrule
\end{tabular}

\begin{description}
    \item[1)] assigns the state of \textit{other} to *this using move semantics
    \item[2)] copies the state of \textit{other} to *this, state (control block)
              is shared
\end{description}

{\bf Parameters}
\begin{description}
    \item[other]   another execution context to assign to this object\\
\end{description}

{\bf Return value}
\begin{description}
    \item[*this]
\end{description}

\subparagraph*{operator()}
jump context of execution\\

\begin{tabular}{ l l }
    \midrule

    \cpp{void operator()() noexcept;} & (1)\\

    \midrule
\end{tabular}

\begin{description}
    \item[1)] resumes the execution context\\
\end{description}

{\bf Exceptions}
\begin{description}
    \item[1)] noexcept specification: \cpp{noexcept}\\
\end{description}

{\bf Notes}
\newline
If an exception leaves this function \cpp{std::terminate()} is called. If this
function returns, \cpp{std::exit(0)} is called.

\subparagraph*{current}
accesses the current active execution context\\

\begin{tabular}{ l l }
    \midrule

    \cpp{static execution_context current();} & (1)\\

    \midrule
\end{tabular}

\begin{description}
    \item[1)] construct a instance of \ectx associated with the current active
              execution context\\
\end{description}

{\bf Notes}
\newline
The current active execution context is thread-specific, e.g. for each thread
(including \main) a execution context is created at start-up.

\uabschnitt{A combined syntax for stackless and stackful context switching?}
A syntax, combining stackless and stackful resumable lambdas, might be possible.
The absence of the stack hint for keyword \resumable could be used to indicate
that a stackless execution context has to be constructed.\\
The disadvatage is that the user and/or the compiler have to ensure that no
context switch is triggered from a nested call stack.
