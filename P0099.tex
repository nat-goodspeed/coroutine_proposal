%//////////////////////////////////////////////////////////////////////////////

\documentclass[paper=A4,pagesize,DIV=15]{scrartcl}

\usepackage[T1]{fontenc}
\usepackage[latin1]{inputenc}
\usepackage[british]{babel}

\usepackage{fixltx2e}
\usepackage{ellipsis}
\usepackage{ragged2e}
\usepackage[final]{microtype}

%\usepackage{lmodern}
\usepackage{palatino}

\usepackage{overcite}
\usepackage{booktabs}
\usepackage{fancyhdr}
\usepackage{listings}
\usepackage{perpage}
\usepackage{rotating}
\usepackage{svg}
\usepackage{tikz}
\usetikzlibrary{arrows,automata}
\usepackage{xcolor}
\usepackage{xspace}
\usepackage[colorlinks=true,
            urlcolor=blue,
            pdftex,
            pdfsubject  = {},
            pdfauthor   = {Oliver Kowalke, Nat Goodspeed},
            pdfkeywords = {C++,execution,context,coroutine,P0099},
            pdftitle    = {A low-level API for stackful context switching}]{hyperref}

%//////////////////////////////////////////////////////////////////////////////

\setlength{\parindent}{0pt} 
\renewcommand\sfdefault{phv}

\makeatletter
    \renewcommand*\l@subsection{\@dottedtocline{2}{0em}{2.3em}}
    \renewcommand*\l@subsection{\@dottedtocline{3}{0em}{3.2em}}
    \renewcommand{\tableofcontents}{%
        \@starttoc{toc}
    }
\makeatother

%\renewcommand{\thesubsection}{\Roman{subsection}}

\newcommand{\pdfimg}[1]{\pdfximage{pics/#1}\pdfrefximage\pdflastximage}
\newcommand{\img}[1]{\mbox{\pdfimg{#1}}}
\newcommand{\imgc}[1]{\begin{center}\img{#1}\end{center}}

\newcommand{\cpp}[1]{
    \lstinline[
        language=C++,
        basicstyle=\ttfamily\color{black},
        keywordstyle=\color{blue},
        commentstyle=\color{green},
        stringstyle=\color{red},
    ] {#1}
}
\newcommand{\cppf}[1]{
    \lstinputlisting[
        language=C++,
        basicstyle=\ttfamily\color{black},
        keywordstyle=\color{blue},
        commentstyle=\color{red},
        stringstyle=\color{green},
    ] {code/#1}
}

\newcommand{\await}{\textit{await}\xspace}
\newcommand{\cblock}{\textit{control-block}\xspace}
\newcommand{\coopmultitasking}{\textit{cooperative multitasking}\xspace}
\newcommand{\corofunction}{\textit{coroutine-function}\xspace}
\newcommand{\resumable}{\textit{resumable}\xspace}
\newcommand{\resumfn}{\textit{resumable function}\xspace}
\newcommand{\resumld}{\textit{resumable lambda}\xspace}
\newcommand{\sfull}{\textit{stackful}\xspace}
\newcommand{\sfcoro}{\textit{stackful} coroutine\xspace}
\newcommand{\sfcoros}{\textit{stackful} coroutines\xspace}
\newcommand{\sless}{\textit{stackless}\xspace}
\newcommand{\slcoro}{\textit{stackless} coroutine\xspace}
\newcommand{\slcoros}{\textit{stackless} coroutines\xspace}
\newcommand{\stack}{\textit{stack}\xspace}
\newcommand{\yield}{\textit{yield}\xspace}

\newcommand{\abschnitt}[1]{
    \addcontentsline{toc}{subsection}{#1}
    \subsection*{#1}
}

\newcommand{\uabschnitt}[1]{
    \addcontentsline{toc}{paragraph}{#1}
    \paragraph*{#1}
}


%//////////////////////////////////////////////////////////////////////////////

\begin{document}

\small
\begin{tabbing}
    Document number: \= P0099R0\\
    Supersedes:      \> N4397\\
    Date:            \> 2015-09-25\\
    Project:         \> WG21, SG1\\
    Author:          \> Oliver Kowalke (oliver.kowalke@gmail.com), Nat Goodspeed (nat@lindenlab.com)\\
\end{tabbing}

\section*{A low-level API for stackful context switching}

%//////////////////////////////////////////////////////////////////////////////

\tableofcontents

%//////////////////////////////////////////////////////////////////////////////

\paragraph*{Revision History}
This document supersedes N4397. It elaborates a low-level API for stackful
context switching.\\
\newline
Changes since N4397:

\begin{itemize}
    \item \cpp{std::execution\_context} presented as pure library facility.
    \item Motivation section added.
    \item \cpp{std::execution\_context} no longer tracks parent relationships.
    \item Removed \cpp{operator bool()} and \cpp{operator!()}.
\end{itemize}

\abschnitt{Abstract}
This paper proposes a \emph{low-level} and \emph{minimal} API for a stackful
execution context, suitable to act as {\bfseries building-block} for
{\bfseries high-level} constructs such as stackful coroutines as well as
cooperative multitasking (aka fibers/user-land threads/green threads).\\
\newline
The most important features are:
\begin{itemize}
    \item first-class object that can be stored in variables or containers
    \item symmetric transfer of execution control, i.e. suspend-by-call -
          enables a richer set of control flows than asymmetric transfer of
          control (i.e. suspend-by-return as described in P0057\cite{P0057} et al.)
    \item benefits of traditional stack management retained
    \item ordinary function calls and returns not affected
    \item working implementation in Boost.Context\cite{bcontext}
\end{itemize}

\abschnitt{Motivation}

This proposal refers to \boostcoroutine as reference implementation - providing
a test suite and examples (some are described in this section).\\
\newline
In order to support a broad range of execution control behaviour the coroutine
types of \scoro and \acoro can be used to \escrecloops, to \escreccomps and for
\coopmultitasking helping to solve problems in a much simpler and more elegant
way than with only a single flow of control.

\subsubsection*{event-driven model}
The event-driven model is a programming paradigm where the flow of a program is
determined by events. The events are generated by multiple independent sources
and an event-dispatcher, waiting on all external sources, triggers callback
functions (event-handlers) whenever one of those events is detected (event-loop).
The application is divided into event selection (detection) and event handling.
\imgc{event_model.pdf}

The resulting applications are highly scalable, flexible, have high
responsiveness and the components are loosely coupled. This makes the event-driven
model suitable for user interface applications, rule-based productions systems
or applications dealing with asynchronous I/O (for instance network servers).


\subsubsection*{event-based asynchronous paradigm}
A classic synchronous console program issues an I/O request (e.g. for user
input or filesystem data) and blocks until the request is complete.
\newline
In contrast, an asynchronous I/O function initiates the physical operation but
immediately returns to its caller, even though the operation is not yet
complete. A program written to leverage this functionality does not block: it
can proceed with other work (including other I/O requests in parallel) while
the original operation is still pending. When the operation completes, the
program is notified. Because asynchronous applications spend less overall time
waiting for operations, they can outperform synchronous programs.
\newline
Events are one of the paradigms for asynchronous execution, but
not all asynchronous systems use events.
Although asynchronous programming can be done using threads, they come with
their own costs:

\begin{itemize}
    \item hard to program (traps for the unwary)
    \item memory requirements are high
    \item large overhead with creation and maintenance of thread state
    \item expensive context switching between threads
\end{itemize}

The event-based asynchronous model avoids those issues:

\begin{itemize}
    \item simpler because of the single stream of instructions
    \item much less expensive context switches
\end{itemize}

The downside of this paradigm consists in a sub-optimal program
structure. A event-driven program is required to split its code into
multiple small callback functions, i.e. the code is organized in a sequence of
small steps that execute intermittently. An algorithm that would usually be
expressed as a hierarchy of functions and loops must be transformed into
callbacks. The complete state has to be stored into a data structure while the
control flow returns to the event-loop.\\
As a consequence, event-driven applications are often tedious and confusing to
write. Each callback introduces a new scope, error callback etc. The
sequential nature of the algorithm is split into multiple callstacks,
making the application hard to debug. Exception handlers are restricted to
local handlers: it is impossible to wrap a sequence of events into a single
try-catch block.
The use of local variables, while/for loops, recursions etc. together with the
event-loop is not possible. The code becomes less expressive.\\
\newline
In the past, code using asio's \asyncops was convoluted by
callback functions.
\cppf{oldasio.cpp}

In this example, a simple echo server, the logic is split into three member
functions - local state (such as data buffer) is moved to member variables.\\
\newline
\boostasio provides with its new \asyncres feature a new
framework combining event-driven model and coroutines, hiding the complexity
of event-driven programming and permitting the style of classic sequential code.
The application is not required to pass callback functions to asynchronous
operations and local state is kept as local variables. Therefore the code
is much easier to read and understand.
Proposal 'N3964: Library Foundations for Asynchronous Operations'\cite{n3964}
describes the usage of coroutines in the context of asynchronous operations.\\
\yieldcontext internally uses \boostcoroutine:
\cppf{coroasio.cpp}

In contrast to the previous example this one gives the impression of sequential
code and local data while using asynchronous operations \asyncread,
\asyncwrite). The algorithm is implemented in one function and error handling is
done by one try-catch block.

\subsubsection*{'same fringe' problem}
The advantages can be seen particularly clearly with the use of a recursive
function, such as traversal of trees.\\
If traversing two different trees in the same deterministic order produces the
same list of leaf nodes, then both trees have the same fringe even if the tree
structure is different.\\
\newline
The same fringe problem could be solved using coroutines by iterating over the
leaf nodes and comparing this sequence via \cpp{std::equal()}. The range of data
values is generated by function \cpp{traverse()} which recursively traverses the
tree and passes each node's data value to its \pushcoro.\\
\pushcoro suspends the recursive computation and transfers the data value to
the main execution context.\\
\pullcoroiterator, created from \pullcoro, steps over those data values and
delivers them to \cpp{std::equal()} for comparison. Each increment of \pullcoroiterator
resumes \cpp{traverse()}. Upon return from \cpp{iterator::operator++()}, either
a new data value is available, or tree traversal is finished (iterator is
invalidated).
\cppf{same_fringe.cpp}

\subsubsection*{\csharp \await}
\csharp contains the two keywords \async and \await. \async introduces a
control flow that involves awaiting asynchronous operations. The compiler
reorganizes the code into a continuation-passing style. \await wraps the rest
of the function after calling \await into a continuation if the asynchronous
operation has not yet completed.\\
The project \awaitemu uses \boostcoroutine for a proof-of-concept
demonstrating the implementation of a full emulation of \csharp \await as a
library extension. Because of stackful coroutines \await is \textbf{not limited}
by "one level" as in \csharp.\\
Evgeny Panasyuk describes the advantages of \boostcoroutine over \await at
\channelnine.
\cppf{await.cpp}

\abschnitt{Background}
At the meeting in Urbana the committee decided to pursue both kinds of
coroutines and encouraged the community to propose a unified syntax.\\
This paper proposes a syntax suiteable for stackless and stackful coroutines
based on the ideas of proposal N4397\cite{N4397}.

\abschnitt{Definitions}

\uabschnitt{execution context}
environment where program logic is evaluated in (CPU registers, stack).

\uabschnitt{control block}
holds a set of registers (callee-saved registers, instruction pointer, stack
pointer) describing the execution context.

\uabschnitt{coroutine}
enables explicit suspend and resume of its progress via additional operations by
preserving execution state and thus provides an enhanced control flow.
Coroutines have following characteristics\cite{N3985}:
\begin{itemize}
    \item values of local data persist between successive context switches
    \item execution is suspended as control leaves coroutine and resumed at
          certain time later
    \item symmetric or asymmetric control transfer-mechanism
    \item first-class object or compiler-internal structure
    \item stackless or stackful
\end{itemize}

\uabschnitt{asymmetric coroutine}
provides two distingt operations for the context switch - one operation to
resume and one operation to suspend the coroutine.\\
An asymmetric coroutine is tightly coupled with its caller, e.g. suspending the
coroutine transferes the execution control back to the point in the code were it
was called. The asymmetric control transfer-mechanism is usually used in the
context of generators.

\uabschnitt{symmetric coroutine}
only one operation to resume/suspend the context is available.\\
A symmetric coroutine does not know its caller, e.g. the execution control can
be transferred to any other symmetric coroutine (must be explicitly specified).
The symmetric control transfer-mechanism is usually used to implement
cooperative multitasking.

\uabschnitt{fiber/user-mode thread}
execute tasks in an cooperative multitasking environment involving a
scheduler. Coroutines and fibers are distinct (N4024\cite{N4024}).

\uabschnitt{resumable function}
N4134\cite{N4134} describes resumable functions as an efficient language
supported mechanism for stackless context switching introducing two new keywords
- \await and \yield. Resumable functions are equivalent to asymmetric
coroutines.\\
Characteristics of resumable functions:
\begin{itemize}
    \item stackless
    \item allocates memory (activation frame) for the body of the resumable
          function to store local data and control block
    \item thight coupling between caller and resumable function (asymmetric
          control transfer-mechanism)
    \item implicit \textit{return}-statement\cite{N4134} (code transformation)
\end{itemize}

\uabschnitt{resumable lambda}
N4244\cite{N4244} describes resumable lambdas as stackless coroutines -
introducing new keywords\\
\resumable, flavours of \yield and \rlthis. Stackless resumable lambdas are
equivalent to asymmetric coroutines.\\
Characteristics of resumable lambdas:
\begin{itemize}
    \item stackless
    \item body of a lambda is available to the compiler at the point
          where the lambda type is defined
    \item compiler can analyse the lambda body to determine what stack
          variables need to be accommodated
    \item space can be made available for the stack variables in exactly the
          same way as is already done for the capture set
    \item thight coupling between caller and resumable lambda (asymmetric
          control transfer-mechanism)
\end{itemize}

\uabschnitt{application stack}
also known as call stack - is a chunk of memory assigned to the stack pointer
and used to store information (for instance local variables) about the active
subroutines. The stack used by function \main grows on demand while the stack
assigned to a thread has a fixed size - usually 1MB (Windows) till 2MB (Linux),
but some platforms use smaller stacks (64KB on HP-UX/PA and 256KB on
HP-UX/Itanium).

\uabschnitt{side stack}
is a call stack that is used in the case of stackful context switching, e.g.
each execution context gets its own stack (assigned to stack pointer). Thus
stack frames of subroutines are allocated on the side stack, the application
stack remains unchanged.

\uabschnitt{activation frame}
is a chunk of memory used by resumable functions to store local (stack)
variables and the control block. Each resumable function is bound to its own
activation frame. The stack pointer remains unchanged, e.g. it still points to
the application stack. Thus stack frames of subroutines are allocated on the 
application stack.\\
Resumable lambdas re-use the storage of the capture set for the same purpose.

\uabschnitt{linked stack}
also known as \textit{split stack}\cite{gccsplit} or
\textit{segmented stack}\cite{llvmseg}, represents a on demand growing stack
with a non-contigous address range.\\
Applications compiled with support for linked stacks can use (link against)
libraries not supporting linked stacks (see GCC's documentation\cite{gccsplit},
chapter 'Backward compatibility').

\abschnitt{Introduction}

This proposal proposes to add \coro to the C++ standard.\\
\newline
In computer science routines are defined as a sequence of operations. The
execution of routines form a parent-child relationship and the child terminates
always before the parent. Coroutines (the term was introduced by Melvin
Conway\cite{Conway1963}) are a generalization of routines (Donald Knuth\cite{Knuth1997}). The
principal difference between coroutines and routines is that a coroutine enables
explicit suspend and resume of their progress via additional operations by
preserving local state. Coroutines support the implementation of components such
as cooperative tasks, iterators, infinite lists and pipes.\\
\newline
\coro is a first class \continuation (can be passed as argument, returned
by procedure and stored in a data structure to be used later).\\
The context switch is cooperative, e.g. the programmer controls when a switch
will happen (the kernel is not involved in the switch between coroutines).\\
\newline
\imgc{sequence.pdf}

\abschnitt{Discussion}
\uabschnitt{Memory requirements}
In the case of a simple toplevel context-function, stackless and stackful coroutines
have the same memory requirements for their activation records.\\
\newline
As described in N4134, the compiler analyses the body of the toplevel context
function and determines the required size and allocates the activation record on
the heap.
\cppf{N4134/fib.cpp}
In the example of calculating Fibonacci numbers using a resumable function, the
compiler reserves space in the activation record for local variables \emph{n},
\emph{a}, \emph{b} and \emph{next}. Those variables are not accessed via the
stack pointer. The processor stack is not used and the stack pointer is not
changed\footnote{Depending on the calling convention, typical \emph{x86} code
stores parameters and return address on the stack. This must be cleaned off by
the time control is returned to the caller. Other architectures/ABIs do
not have this requirement}.\\
\newline
Each stackful coroutine owns a side stack (assigned to the stack pointer). The
activation record of the toplevel context function is stored on the side
stack, thus the original stack remains unchanged.
\cppf{N4397/fib.cpp}
In the case of calculating Fibonacci numbers using a stackful execution
context, the compiler stores the local variables \emph{n}, \emph{a},
\emph{b} and \emph{next} on the side stack. The local variables are accessed
via the stack pointer.\\
\newline
The memory requirements for both types of coroutines are equal.

\uabschnitt{Calling subroutines}
The advantage of stackless coroutines is that they reuse the same linear
processor stack for stack frames for called subroutines. The advantage of
stackful context switching is that it permits {\bfseries suspending from
nested calls}.\\
\newline
If a traditional function (not another resumable function) is invoked inside
the body of a resumable function, then the activation record belonging to the
traditional function is allocated on the processor stack (so it is called a
stack frame). As a consequence, stack frames of called functions must be
removed from the processor stack before the resumable function yields back to
its caller.\\
In other words: the calling convention of the ABI dictates that, after the
resumable function returns (suspends), the stack pointer contains the same
address as before the resumable function was entered.\\
Hence a yield from nested call is {\bfseries not permitted} -- unless every
called function down to the yield point is also a resumable function.\\
The benefit of stackless coroutines consists in reusing the processor stack
for called subroutines: no separate stack memory need be allocated.\\
\newline
Of course even a stackless resumable function might fail if its called
functions exhaust the available stack.\\
\newline
In stackful context switching, each execution context owns a
distinct side stack which is assigned to the stack pointer (thus the stack
pointer must be exchanged during each context switch).\\
All activation records (stack frames) of subroutines are placed on the side
stack. Hence each stackful execution context requires enough memory to hold
the stack frames of the longest call chain of subroutines. Therefore, to
support calling subroutines, stackful context switching has a higher memory
footprint than resumable functions.\\
On the other hand, it is beneficial to use side stacks because the stack
frames of active subroutines remain intact while the execution context is
suspended. This is the reason why stackful context switching permits
{\bfseries yielding from nested calls}.
\cppf{N4397/simple.cpp}
Variable \emph{l1} is passed to stackful execution context \emph{l2} via
the capture list of \emph{l2}. The stack frames of \cpp{std::printf()} are
allocated on the side stack owned by \emph{l2}.

\uabschnitt{Asymmetric vs. symmetric}
As a building block for user-mode threads, symmetric control transfer is more
efficient than the asymmetric mechanism.\\
\newline
Resumable functions (as proposed in N4134\cite{N4134}) provide two operations
for context switching: asymmetric control transfer. The caller and the
resumable function are coupled, that is, a resumable function can only jump back
to the point in the code where it was entered.
\imgc{asymm.pdf}
For N resumable functions \emph{2N} context switches are required.\\
\newline
This is sufficient in the case of generators, but in the context of cooperative
multitasking it is inefficient.
In contrast to resumable functions, the proposed stackful execution context
(\ectx) provides only one operation to resume/suspend the context (\ectxop).
Control is directly transferred from one execution context to
another (symmetric control transfer) - no jump back to the caller. In addition
to supporting generators, this enables an efficient implementation of
cooperative multitasking: no additional context switch back to caller,
direct context switch to next task.\\
The next execution context must be explicitly specified.\\
\newline
\imgc{symm.pdf}
Resuming N instances of \ectx takes \emph{N+1} context switches.

\uabschnitt{Passing data}
Because of the asymmetric resume/suspend operations of N4134, the proposal
applies well to generator examples, e.g. returning data from the resumable
function.\\
\newline
Passing data into the body of resumable functions (N4123) requires helper
classes like \channel.
\cppf{N4134/channel.cpp}
In this case, the way data are passed into the body is not intuitive and
introduces some problems.\\
\newline
Experience with \cpp{execution_context} from Boost.Context\cite{bcontext}
turned up a common pattern: to pass a lambda to the \cpp{execution_context},
using its capture list to transfer data into and to return data from the
body. Exceptions are transferred via \excpt and parameters (input/output) are
accessed via captured references or pointers.
\cppf{N4397/param.cpp}
Class \cpp{X} (rudimentary coroutine) demonstrates how input and output
parameters are transferred between contexts. Member variable
\emph{callee\_} represents a new execution context and captures the
\emph{this}-pointer of \cpp{X}. The body converts a integer variable (input)
into a string (output). Any exception thrown by the conversion is transported
via member variable \emph{exptr\_}.

\uabschnitt{Stack strategies}
For stackful coroutines two strategies are typical: a contiguous, fixed-size
stack (as used by threads), or a linked stack (grows on demand).\\
The advantage of a fixed-size stack is the fast allocation/deallocation of 
activation records. A disadvantage is that the required stacksize has to
be guessed.\\
The benefit of using a linked stack is that only the initial size of the stack
is required. The stack itself grows on demand, by means of an overflow handler.
The performance penalty is low. The disadvantage is that
code executed inside a stackful coroutine must be rebuilt for this
purpose. In the case of GCC's split stacks, special compiler/linker flags must
be specified - no changes to source code are required.\\
When calling a library function not compiled for linked stacks (expecting a
traditional contiguous stack), GCC's implementation uses link-time code
generation to change the instructions in the caller. The effect is that a
reasonably large contiguous stack chunk is temporarily linked in to handle the
deepest expected chain of traditional function stack frames.

\abschnitt{Design}
Class \ectx provides a {\bfseries small, basic API} on which to build {\bfseries
higher-level APIs} such as stackful coroutines (N3985\cite{N3985}) and user-mode
threads (such as Boost.Fiber\cite{bfiber}).


\uabschnitt{Suspend-by-call}
\ectxop preserves the CPU register set\footnote{defined by ABI's calling
convention}: the content of those registers is pushed at the end of the stack
of the current context (at the current stack-pointer). Then \op restores the
stack-pointer register stored in \cpp{*this} and pops the CPU register set
from the newly-restored stack.
Because the context state is preserved on the context's stack, a \ectx
instance need only store the stack-pointer register.


\uabschnitt{Call semantics}
When \ectxop is called, a new instance of \ectx is synthesized representing
the current state of the running context (e.g. the stack-pointer). This new
instance is passed to the resumed context. On initial entry, it is passed as
the first argument to the top-level function. On every subsequent resumption,
it is returned by the suspended \op call.

On completion of a successful context switch, the
\ectx instance on which \op was called is invalidated. The data member from
which the stack pointer was just restored is set to \cpp{nullptr}.

At most one instance of \ectx can represent a given execution context. The
currently-running execution context is not represented by \emph{any} \ectx
instance. Only when \op is called on some \emph{other} \ectx instance is the
state of the running execution context captured in a synthesized \ectx
instance.

As mentioned in the section below on stack
destruction, \cpp{\~execution\_context<>()} on a suspended (not terminated)
instance destroys the stack managed by that instance. Thus, the stack must be
managed by only one \ectx instance.\footnote{An earlier design used reference
counting, but that subverts the intended role of this facility as an extremely
fast substrate for higher-level libraries.}

Because of the symmetric context switching (only one operation transfers
control), the target execution context must be explicitly specified.

\uabschnitt{std::execution\_context<void>}
With \cpp{execution\_context<void>} no data will be transferred, only the
context switch is executed.
\cppf{passing_void}
\cpp{ctx1()} resumes \cpp{ctx1}, that is, control enters the lambda passed to
the constructor of \cpp{ctx1}. Argument \cpp{ctx2} represents the previous
context: the context that was suspended by the call to \cpp{ctx1()}. When the
lambda returns \cpp{ctx2}, context \cpp{ctx1} will be terminated while the
context represented by \cpp{ctx2} is resumed, hence control returns
from \cpp{ctx1()}.\\

\uabschnitt{Passing data}
When you construct a \ectx with template arguments other than \cpp{void}, the
function or lambda that initializes that instance must accept parameters\\
\cpp{(std::execution\_context<args...>, args...)}, where \cpp{args...} here
represents any list of arguments other than \cpp{void}.

The initial \ectx argument is synthesized by \op. All other arguments must be
passed explicitly to \op.

The first call to \op with those arguments populates the parameter list for
the newly-entered function or lambda.

That function or lambda switches context back to the original context by
calling the passed\\
\ectxop, passing appropriate arguments.

The \emph{original} context's call to \op returns
a \cpp{std::tuple<std::execution\_context<args...>, args...>}. The
returned \ectx is a synthesized instance representing the context that just
suspended. The rest of the \cpp{args...} are as passed by that context to \op.

So, for instance:
\cppf{passing_single}
The \cpp{ctx1(i)} call at (a) enters the lambda in context \cpp{ctx1} with
argument \cpp{j=1}, as shown by the output at (b). The
expression \cpp{ctx2(j+1)} at (c) resumes the original context (represented
within the lambda by \cpp{ctx2}) and transfers back an integer of \cpp{j+1}.
On return from \cpp{ctx1(i)}, the assignment at (d) sets \cpp{i} to \cpp{j+1},
or 2.

The assignment at (d) illustrates a recommended idiom: since the call to \op
at (a) has invalidated \cpp{ctx1}, it should be replaced by the
newly-synthesized \ectx instance returned by \op.

To continue the example:
\cppf{passing_single_continued}
The call to \cpp{ctx1(i)} at (e) (the \emph{updated} \cpp{ctx1}) resumes
the \cpp{ctx1} lambda, returning from the \cpp{ctx2()} call at (c) and
executing the assignment at (f). Here, too, we replace the \ectx
instance \cpp{ctx2} invalidated by the \op call at (c) with the new instance
returned by that same \op call. Moreover, we replace \cpp{j} with the value
passed by the call at (e).

Finally the lambda returns (the updated) \cpp{ctx2} at (g), terminating its
context.

Since the updated \cpp{ctx2} represents the context suspended by the call at
(e), control returns to the assignment at (h). Once again we replace the
invalidated \cpp{ctx1} with the one returned by \op.

However, since context \cpp{ctx1} has now terminated, the updated \cpp{ctx1}
is \emph{not-a-context}. Its \cpp{operator bool()} returns \cpp{false};
its \cpp{operator\!()} returns \cpp{true}.

This is important, since in that case the values of any remaining fields of
the returned \cpp{std::tuple} are indeterminate.

It may seem tricky to keep track of which \ectx instance is currently
``live,'' representing the state of the suspended context. Please bear in
mind that this facility is intended as a high-performance foundation for
higher-level libraries. It is not intended to be directly consumed by
applications.\\
\newline
We can extend the example to multiple arguments.
\cppf{passing_multiple}
\op accepts the parameters specified by \ectx's template parameters. It
returns a \cpp{std::tuple} of that \ectx specialization, prepended to those
types.


\uabschnitt{Top-level thread functions}
\main as well as the \emph{entry-function} of a thread can be represented by an
execution context. That \ectx instance is synthesized when the running context
suspends, and is passed into the newly-resumed context.


\uabschnitt{\ectx and std::thread}
The context represented by a \ectx instance is necessarily suspended. It is
valid to resume a \ectx instance on any thread -- \emph{except} that you must
not attempt to resume a \ectx instance representing \main, or
the \emph{entry-function} of some other \cpp{std::thread}, on any thread other
than its own.

\ectx provides a method to test for this.
If \cpp{std::execution\_context<>::any\_thread()} returns \cpp{false}, it is
only valid to resume that \ectx instance on the thread on which it was
initially launched.


\uabschnitt{Termination}
When you explicitly construct a particular \cpp{std::execution\_context<args...>}
specialization, passing its constructor a function or lambda, that callable
must accept that same\\
\cpp{std::execution\_context<args...>} specialization as
its first parameter. It must return that
same \cpp{std::execution\_context<args...>} specialization as well.

When that toplevel callable returns a \ectx instance, the running context is
voluntarily terminated. Control switches to the context indicated by the
returned \ectx instance.

If the toplevel callable returns the same \ectx instance it was originally
passed (or rather, the most recently updated instance returned from the
original instance's \op), control returns to the context that originally
resumed the running callable. However, the callable may return (switch to)
any reachable valid \ectx instance with the correct type signature.


\uabschnitt{Exceptions}
If an uncaught exception escapes from the toplevel context function,
\cpp{std::terminate} is called.


\uabschnitt{Executing function on top of a context}
Sometimes it is useful to execute a new function (for instance, to trigger
unwinding the stack) on top of a suspended context. For this purpose you may
pass to \ectxop:

\begin{itemize}
  \item the special argument \cpp{invoke\_ontop\_arg}
  \item the function to execute
  \item any additional arguments required by the \ectx specialization.
\end{itemize}

The function passed in this case must accept as parameters the \ectx
specialization for that context plus any arguments specified by that
specialization. It must return a tuple of that \ectx specialization plus the
same set of arguments.\footnote{But in the case
of \cpp{std::execution\_context<void>}, the return type is
simply \cpp{std::execution\_context<void>}.}

For purposes of discussion, consider two \cpp{std::execution\_context<void>}
instances: \cpp{mainctx} and \cpp{fctx}. Suppose that code running
on the program's main context instantiates \cpp{fctx} with function\\
\cpp{f(std::execution\_context<void> mainctx)} and calls \cpp{fctx()}, thereby
entering \cpp{f()}. This is the point at which \cpp{mainctx} is synthesized
and passed into \cpp{f()}.

Suppose further that after doing some work, \cpp{f()} calls \cpp{mainctx()},
thereby switching context back to the main context. \cpp{f()} remains
suspended in the call to \cpp{mainctx.operator()()}.

At this point the main context calls \cpp{fctx(std::invoke\_ontop\_arg, g);}
where \cpp{g()} is declared as:
\newline
\cpp{std::execution\_context<void> g(std::execution\_context<void> gmctx);}
\newline
\cpp{g()} is entered in the context of \cpp{f()}. It is as if \cpp{f()}'s call
to \cpp{mainctx.operator()()} directly called \cpp{g()}.

However, as usual, the \cpp{fctx.operator()()} call synthesizes a \ectx
instance representing the main context and passes it to \cpp{g()}
as \cpp{gmctx}.

Function \cpp{g()} has the same range of possibilities as any function called
on \cpp{f()}'s context. It can context-switch back to the main context, or to
any other reachable context. Its special invocation only matters when control
leaves it in either of two ways:

\begin{enumerate}
  \item If \cpp{g()} throws an exception, that exception unwinds all previous
  stack entries in that context (such as \cpp{f()}'s) as well, back to a
  matching \cpp{catch} clause.
  \item If \cpp{g()} returns, its return value becomes the value returned
  by \cpp{f()}'s suspended \cpp{mainctx.operator()()} call. This is
  why \cpp{g()}'s return type must be the same as that of \op, rather than
  that of an ordinary toplevel context function.
\end{enumerate}

Consider the following example:

\cppf{ontop}

Control passes from (a) to (b) to (c), and so on.

The \cpp{ctx(std::invoke\_ontop\_arg, f2, data+1)} call at (l) passes control
to \cpp{f2()} on the context originally created for \cpp{f1()}.

The \cpp{return} statement at (n) causes the \op call at (i) to return,
executing the assignment at (o). The \cpp{std::tuple} returned by \cpp{f2()}
is directly returned to that assignment at (o).

So in this example, the call at (l) synthesizes a \ectx instance representing
the main context and passes it to \cpp{f2()} as \cpp{ctx}. \cpp{f2()} returns
that \cpp{ctx} instance, which is received by \cpp{f1()} and assigned
to \emph{its} \cpp{ctx} variable. Finally, \cpp{f1()} returns its
own \cpp{ctx} variable, switching back to the main context.


\uabschnitt{Stack destruction}
On construction of \cpp{execution\_context} a stack is allocated. If the toplevel
context-function returns, the stack will be destroyed. If the context-function
has not yet returned and the destructor of a valid \cpp{execution\_context}
instance (\cpp{execution\_context::operator bool()} returns \cpp{true}) is
called, the stack will be unwound and destroyed.\footnote{An implementation is
free to unwind the stack without throwing an exception. However, if an
exception is thrown, it should be named \cpp{std::execution\_context\_unwind}.
Portable consumer code \emph{must} permit \cpp{std::execution\_context\_unwind}
exceptions to propagate, even if all other exceptions are caught
with \cpp{catch (...)}.}


\uabschnitt{API}
declaration of class \ectx
\cppf{ec}
\paragraph*{member functions}
\subparagraph*{(constructor)}
constructs new execution context\\

\begin{tabular}{ l l }
    \midrule

    \cpp{execution\_context() noexcept} & (1)\\

    \midrule

    \cpp{template<typename StackAlloc, typename Fn, typename ... Args>}\\
    \cpp{execution\_context(std::allocator\_arg\_t, StackAlloc salloc,}\\
    \cpp{                   Fn&& fn, Args&& ... args)} & (2)\\

    \midrule

    \cpp{template<typename Fn, typename ... Args>}\\
    \cpp{explicit execution\_context(Fn&& fn, Args&& ... args)} & (3)\\

    \midrule

    \cpp{execution\_context(execution\_context&& other)} & (4)\\

    \midrule

    \cpp{execution\_context(const execution\_context& other)=delete} & (5)\\

    \midrule
\end{tabular}

\begin{description}
    \item[1)] this constructor represents \emph{not-a-context}
    \item[2)] this constructor takes (e.g.) a lambda as argument, stack is
              constructed using \emph{salloc}
    \item[3)] takes (e.g.) lambda as argument,
              stack is constructed using either \cpp{fixedsize}
              or \cpp{segmented}. An implementation may infer which of these
              best suits the code in \cpp{fn}. If it cannot
              infer, \cpp{fixedsize} will be used.
    \item[4)] moves underlying capture record to new \ectx
    \item[5)] copy constructor deleted
\end{description}

{\bfseries Notes}
\newline
When an \ectx is constructed using either of the constructors accepting
\cpp{fn}, control is \emph{not} immediately passed to \cpp{fn}. Resuming
(entering) \cpp{fn} is performed by calling \cpp{operator()()} on the new
\ectx instance.\\

\subparagraph*{(destructor)}
destroys an execution context\\

\begin{tabular}{ l l }
    \midrule

    \cpp{\~execution\_context()} & (1)\\

    \midrule
\end{tabular}

\begin{description}
    \item[1)] destroys a \ectx. If associated with a context of execution and
              holds the last reference to the internal capture record, then the
              context of execution is destroyed too. Specifically, the stack is
              unwound.\\
\end{description}

\subparagraph*{operator=}
copies/moves the context object\\

\begin{tabular}{ l l }
    \midrule

    \cpp{execution\_context& operator=(execution\_context&& other)} & (1)\\

    \midrule

    \cpp{execution\_context& operator=(const execution\_context& other)=delete} & (2)\\

    \midrule
\end{tabular}

\begin{description}
    \item[1)] assigns the state of \emph{other} to \emph{*this} using move semantics
    \item[2)] copy assignment operator deleted
\end{description}

{\bfseries Parameters}
\begin{description}
    \item[other]   another execution context to assign to this object\\
\end{description}

{\bfseries Return value}
\begin{description}
    \item[*this]
\end{description}

\subparagraph*{operator()}
resume context of execution\\

\begin{tabular}{ l l }
    \midrule

    \cpp{std::tuple<execution\_context, Args ...> operator()(Args ... args)} & (1)\\

    \midrule

    \cpp{execution\_context<void> operator()()} & (2) \\

    \midrule

    \cpp{template<typename Fn>}\\
    \cpp{std::tuple<execution\_context, Args ...>}\\
    \cpp{operator()(invoke\_ontop\_arg\_t, Fn&& fn, Args ... args)} & (3)\\

    \midrule

    \cpp{template<typename Fn>}\\
    \cpp{execution\_context<void>}\\
    \cpp{operator()(invoke\_ontop\_arg\_t, Fn&& fn)} & (4)\\

    \midrule
\end{tabular}

\begin{description}
    \item[1)] suspends the active context, resumes the execution context
    \item[2)] specialization of (1) for \cpp{execution\_context<void>}
    \item[3)] suspends the active context, resumes the execution context but
        executes \cpp{fn(args ...)} in the resumed context (e.g. on top of the
        last stack frame)
    \item[4)] specialization of (3) for \cpp{execution\_context<void>}
\end{description}

{\bfseries Parameters}
\begin{description}
    \item[... args] passed to current context  returned by the most recent call
                    to \cpp{execution\_context::operator()}\\
\end{description}

{\bfseries Return value}
\begin{description}
    \item[tuple]    of \cpp{execution\_context} and returned arguments passed to
                    the most recent call to\\ \cpp{execution\_context::operator()},
                    if any and a \cpp{execution\_context} representing the
                    context that has been suspended\\
\end{description}

{\bfseries Exceptions}
\begin{description}
    \item[1)] calls \cpp{std::terminate} if an exception escapes toplevel \cpp{fn}\\
\end{description}

{\bfseries Notes}
\newline
The \emph{prologue} preserves the execution context of the calling context as
well as stack parts like \emph{parameter list} and \emph{return
address}.\footnote{required only by some x86 ABIs} Those data are restored by
the \emph{epilogue} if the calling context is resumed.
\newline
A suspended \cpp{execution\_context} can be destroyed. Its resources will be
cleaned up at that time.
\newline
The returned \cpp{execution\_context} indicates if the suspended context has
terminated (return from context-function) via \cpp{bool operator()}.
If the returned \cpp{execution\_context} has terminated no data are transferred
in the returned tuple.


\subparagraph*{operator bool}
test context if valid\\

\begin{tabular}{ l l }
    \midrule

    \cpp{explicit operator bool() const noexcept} & (1)\\

    \midrule
\end{tabular}

\begin{description}
    \item[1)] returns \cpp{true} if \cpp{*this} refers to an valid \ectx,
              \cpp{false}\xspace otherwise
\end{description}

\subparagraph*{operator!}
test context if not valid\\

\begin{tabular}{ l l }
    \midrule

    \cpp{bool operator\!() const noexcept} & (1)\\

    \midrule
\end{tabular}

\begin{description}
    \item[1)] returns \cpp{true} if \cpp{*this} refers not to an valid \ectx,
              \cpp{false}\xspace otherwise
\end{description}


\uabschnitt{Stack allocators}
are described in P0099R0\cite{P0099R0}.

\abschnitt{Acknowledgements}
The authors would especially like to thank Christopher Kohlhoff and Ville Voutilainen for
encouragement and valuable suggestions.


%//////////////////////////////////////////////////////////////////////////////

\addcontentsline{toc}{subsection}{References}
\begin{thebibliography}{99}

    \bibitem{N3985}
        \href{http://www.open-std.org/jtc1/sc22/wg21/docs/papers/2014/n3985.pdf}
        {N3985: A proposal to add coroutines to the C++ standard library}

    \bibitem{N4232}
        \href{http://www.open-std.org/jtc1/sc22/wg21/docs/papers/2014/n4232.pdf}
        {N4232: Stackful Coroutines and Stackless Resumable Functions}

    \bibitem{N4397}
        \href{http://www.open-std.org/jtc1/sc22/wg21/docs/papers/2015/n4397.pdf}
        {N4397: A low-level API for stackful coroutines}

    \bibitem{N4453}
        \href{http://www.open-std.org/jtc1/sc22/wg21/docs/papers/2015/n4453.pdf}
        {N4453: Resumable Expressions}

    \bibitem{P0057}
        \href{http://www.open-std.org/jtc1/sc22/wg21/docs/papers/2016/p0057r5.pdf}
        {P0057R5: Wording for Coroutines}

    \bibitem{P0099R0}
        \href{http://www.open-std.org/jtc1/sc22/wg21/docs/papers/2015/p0099r0.pdf}
        {P0099R0: A low-level API for stackful context switching}

    \bibitem{P0112}
        \href{http://www.open-std.org/jtc1/sc22/wg21/docs/papers/2015/p0112r0.html}
        {P0112R0: Networking Library Proposal (Revision 6)}

    \bibitem{P0158}
        \href{http://www.open-std.org/jtc1/sc22/wg21/docs/papers/2015/p0158r0.html}
        {P0158R0: Coroutines belong in a TS}

    \bibitem{gccsplit}
        \href{http://gcc.gnu.org/wiki/SplitStacks}
        {Split Stacks / GCC}

    \bibitem{basio}
        \href{http://www.boost.org/doc/libs/release/doc/html/boost\_asio.html}
        {Library \emph{Boost.Asio}}

    \bibitem{basiocoro}
        \href{http://www.boost.org/doc/libs/release/doc/html/boost_asio/reference/coroutine.html}
        {\emph{Boost.Asio} Coroutines}

    \bibitem{bcontext}
        \href{http://www.boost.org/doc/libs/release/libs/context/doc/html/index.html}
        {Library \emph{Boost.Context}}

    \bibitem{bcoroutine2}
        \href{http://www.boost.org/doc/libs/release/libs/coroutine2/doc/html/index.html}
        {Library \emph{Boost.Coroutine2}}

    \bibitem{bfiber}
        \href{http://www.boost.org/doc/libs/release/libs/fiber/doc/html/index.html}
        {Library \emph{Boost.Fiber}}

    \bibitem{boostcon}
        \href{https://www.youtube.com/watch?v=gcNphOWuUb0}
        {C++Now 2016: Nat Goodspeed, \emph{The Fiber Library}}

    \bibitem{cppcon}
        \href{https://www.youtube.com/watch?v=e-NUmyBou8Q}
        {CppCon 2016: Nat Goodspeed, \emph{Elegant Asynchronous Code}}

\end{thebibliography}


%//////////////////////////////////////////////////////////////////////////////

\abschnitt{A. \textit{jump}-operation for SYSV ABI on x86\_64}\label{appendix}

The assembler code (from \boostcontext) shows what the \textit{jump}-operation
might look like for SYSV ABI on x86\_64.
\asmf{jump.S}
Register \asm{rdi} contains a reference to the \cblock of the current execution
context \textit{X} and register \asm{rsi} points to the \cblock of the execution
context \textit{Y} which has to be resumed.\\
\newline
In lines 2-7 the content of the current non-volatile registers are stored in the
\cblock of \textit{X}.\\
Line 9 calculates the stack-pointer writes it to the \cblock in line 10.\\
Lines 11 and 12 do the same for the return address (will be assigned to the
instruction-pointer \asm{rip}).\\
\newline
The next block (lines 14-19) restore the content of non-volatile register for
the execution context \textit{Y}.\\
The stack-pointer is restored in line 21.\\
Line 22 moves the address of the instruction which should be executed in
\textit{Y} to register \asm{rcx}.\\
\newline
Lines 24 and 25 allow to transfer data (as return value in \textit{Y}) between
context jumps.\\
\newline
The next line transfers execution control (\textit{branch-and-link}) to
\textit{Y} by executing the instruction stored in \asm{rcx}.


%//////////////////////////////////////////////////////////////////////////////

\end{document}
